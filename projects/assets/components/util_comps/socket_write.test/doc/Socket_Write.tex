\documentclass{article}
\iffalse
This file is protected by Copyright. Please refer to the COPYRIGHT file
distributed with this source distribution.

This file is part of OpenCPI <http://www.opencpi.org>

OpenCPI is free software: you can redistribute it and/or modify it under the
terms of the GNU Lesser General Public License as published by the Free Software
Foundation, either version 3 of the License, or (at your option) any later
version.

OpenCPI is distributed in the hope that it will be useful, but WITHOUT ANY
WARRANTY; without even the implied warranty of MERCHANTABILITY or FITNESS FOR A
PARTICULAR PURPOSE. See the GNU Lesser General Public License for more details.

You should have received a copy of the GNU Lesser General Public License along
with this program. If not, see <http://www.gnu.org/licenses/>.
\fi

\author{} % Force author to be blank
%----------------------------------------------------------------------------------------
% Paper size, orientation and margins
%----------------------------------------------------------------------------------------
\usepackage{geometry}
\geometry{
  letterpaper,      % paper type
  portrait,         % text direction
  left=.75in,       % left margin
  top=.75in,        % top margin
  right=.75in,      % right margin
  bottom=.75in      % bottom margin
 }
%----------------------------------------------------------------------------------------
% Header/Footer
%----------------------------------------------------------------------------------------
\usepackage{fancyhdr} \pagestyle{fancy} % required for fancy headers
\renewcommand{\headrulewidth}{0.5pt}
\renewcommand{\footrulewidth}{0.5pt}
%----------------------------------------------------------------------------------------
% Appendix packages
%----------------------------------------------------------------------------------------
\usepackage[toc,page]{appendix}
%----------------------------------------------------------------------------------------
% Defined Commands & Renamed Commands
%----------------------------------------------------------------------------------------
\renewcommand{\contentsname}{Table of Contents}
\renewcommand{\listfigurename}{List of Figures}
\renewcommand{\listtablename}{List of Tables}
\newcommand{\todo}[1]{\textcolor{red}{TODO: #1}\PackageWarning{TODO:}{#1}} % To do notes
\newcommand{\code}[1]{\texttt{#1}} % For inline code snippet or command line
%----------------------------------------------------------------------------------------
% Various pacakges
%----------------------------------------------------------------------------------------
\usepackage{hyperref} % for linking urls and lists
\usepackage{graphicx} % for including pictures by file
\usepackage{listings} % for coding language styles
\usepackage{rotating} % for sideways table
\usepackage{pifont}   % for sideways table
\usepackage{pdflscape} % for landscape view
%----------------------------------------------------------------------------------------
% Table packages
%----------------------------------------------------------------------------------------
\usepackage{longtable} % for long possibly multi-page tables
\usepackage{tabularx} % c=center,l=left,r=right,X=fill
\usepackage{float}
\floatstyle{plaintop}
\usepackage[tableposition=top]{caption}
\newcolumntype{P}[1]{>{\centering\arraybackslash}p{#1}}
\newcolumntype{M}[1]{>{\centering\arraybackslash}m{#1}}
%----------------------------------------------------------------------------------------
% Block Diagram / FSM Drawings
%----------------------------------------------------------------------------------------
\usepackage{tikz}
\usetikzlibrary{shapes,arrows,fit,positioning}
\usetikzlibrary{automata} % used for the fsm
\usetikzlibrary{calc} % For duplicating clients
\usepgfmodule{oo} % To define a client box
%----------------------------------------------------------------------------------------
% Colors Used
%----------------------------------------------------------------------------------------
\usepackage{colortbl}
\definecolor{blue}{rgb}{.7,.8,.9}
\definecolor{ceruleanblue}{rgb}{0.16, 0.32, 0.75}
\definecolor{drkgreen}{rgb}{0,0.6,0}
\definecolor{deepmagenta}{rgb}{0.8, 0.0, 0.8}
\definecolor{cyan}{rgb}{0.0,0.6,0.6}
\definecolor{maroon}{rgb}{0.5,0,0}
%----------------------------------------------------------------------------------------
% Update the docTitle and docVersion per document
%----------------------------------------------------------------------------------------
\def\docTitle{Component Data Sheet}
\def\docVersion{1.4}
%----------------------------------------------------------------------------------------
\date{Version \docVersion} % Force date to be blank and override date with version
\title{\docTitle}
\lhead{\small{\docTitle}}

\def\comp{socket\_write}
\edef\ecomp{socket_write}
\def\Comp{Socket Write}
\graphicspath{ {figures/} }

\begin{document}

\section*{Summary - \Comp}
\begin{tabular}{|c|M{13.5cm}|}
  \hline
  \rowcolor{blue}
                    &                            \\
  \hline
  Name              & \comp                      \\
  \hline
  Worker Type       & Application (Testing)      \\
  \hline
  Version           & v\docVersion \\
  \hline
  Release Date      & October 2018 \\
  \hline
  Component Library & ocpi.assets.util\_comps     \\
  \hline
  Workers           & \comp.rcc                  \\
  \hline
  Tested Platforms  & linux-c7-x86\_64, linux-x13\_3-arm \\
  \hline
\end{tabular}
\section*{Functionality}
\begin{flushleft}
  The Socket Write component forwards all incoming data to a TCP port \textit{or} acts as a demultiplexer/router by parsing any protocol and routing different opcodes to various output ports.\par\medskip
  The serving/listening TCP ports, along with various ways to determine the Worker's ``done'' status, are extremely configurable using Properties.\par\medskip
\end{flushleft}
\begin{center}
\framebox{\parbox{0.8\linewidth}{\centering This Component provides \textit{minimal} error checking and is \textbf{not recommended for production use}, but is only intended for prototyping and testing of other Components.}}
\end{center}

\section*{Block Diagrams}
\subsection*{Top level}
\makeatletter
\newcommand{\gettikzxy}[3]{%
  \tikz@scan@one@point\pgfutil@firstofone#1\relax
  \edef#2{\the\pgf@x}%
  \edef#3{\the\pgf@y}%
}
\makeatother
\pgfooclass{clientbox}{ % This is the class clientbox
    \method clientbox() { % The clientbox
    }
    \method apply(#1,#2,#3) { % Causes the clientbox to be shown at coordinate (#1,#2) and named #3
        \node[rectangle,ultra thick,draw=black,fill=blue] at (#1,#2) (#3) {Remote Client};
    }
}
\pgfoonew \myclient=new clientbox()
\begin{center}
  \begin{tikzpicture}[% List of styles applied to all, to override specify on a case-by-case
      every node/.style={
        align=center,      % use this so that the "\\" for line break works
        minimum size=2cm  % creates space above and below text in rectangle
      },
      every edge/.style={draw,thick}
    ]
    \node[rectangle,ultra thick,draw=black,fill=blue](R2){\Comp};
    \node[rectangle,draw=white,fill=white](R3)[left= of R2]{``in'' \\ Data to be streamed\\ to output port(s)};
    \node[rectangle,draw=white,fill=white,minimum size=1.0cm](R5)[above= of R2]{See property table};
    \node[rectangle,draw=white,fill=white](R4)[right= of R2]{``out'' \\ Data to be streamed\\ to follow-on Component};
    \node[rectangle,draw=white,fill=white](placeholder)[below= of R2]{};
    \path[->]
    (R3)edge []  node [] {} (R2)
    (R2)edge []  node [] {} (R5)
    (R5)edge []  node [] {} (R2)
    (R2)edge []  node [] {} (R4)
    ;
    \gettikzxy{(placeholder)}{\rx}{\ry}
    \foreach [evaluate=\s as \ss using {6-\s}] \s in {1,...,5} {
      \myclient.apply(-10 + \rx + 6*\s,\ry - 6*\s,C\s);
      \path[<->]($(R2.south) + (20-6*\ss pt,0)$) edge [] node [] {} (C\s);
    }
  \end{tikzpicture}
\end{center}

\section*{Source Dependencies}
\subsection*{\comp.rcc}
\def\fpath{ocpiassets/components/util\_comps/\comp.rcc/}
\begin{itemize}
  \item \fpath\comp.cc
  \foreach \f in {
  connection.cpp,
  connection.hpp,
  connection\_manager.cpp,
  connection\_manager.hpp,
  outbound.hpp,
  server.cpp,
  server.hpp
  } {
  \item \fpath ext\_src/\f
  }
  \item \fpath asio/*  \footnote{Externally provided \href{http://think-async.com/Asio/}{ASIO library} for asynchronous IO with C++, with OpenCPI-specific build system}
\end{itemize}

\subsubsection*{\comp.rcc Compilation}
Because OpenCPI maintains backwards compatibility with older compilers, a fully-compliant C++11 environment is \textit{not} required. However, the workaround for non-C++11-compliance is that the ASIO library has dependencies on the Boost\footnote{\url{http://www.boost.org/}} library, \textit{e.g.} on CentOS~6, it requires the \verb+boost-devel+, \verb|boost-thread|, and \verb|boost-static| RPMs.\medskip

To build this component targeting a non-x86 platform, the vendor must provide the appropriate \verb+boost_system+ and \verb+boost_thread+ \textit{static} library files. The Worker's build system will attempt to find them using the \verb+locate+ command in a subdirectory that has \verb+${OCPI_CROSS_HOST}+ within the path.\medskip

See the enclosed \verb+README+ file for more information, including how to add new platforms.

\begin{landscape}
  \section*{Component Spec Properties}
  \begin{minipage}{\textwidth}
    \renewcommand*\footnoterule{} % Remove separator line from footnote
    \renewcommand{\thempfootnote}{\arabic{mpfootnote}} % Use Arabic numbers (or can't reuse)
  \begin{scriptsize}
    \begin{tabular}{|p{3.2cm}|p{1.5cm}|c|c|c|c|c|p{7cm}|}
      \hline
      \rowcolor{blue}
      Name                          & Type      & SequenceLength & ArrayDimensions & Accessibility      & Valid Range & Default       & Usage \\
      \hline
      \verb+outSocket+\footnote{This structure is only read at Component START to configure.} & Struct & - & - & Writable, Readable & - & - & TCP socket(s) to use for listening \\
      \hline
      \verb+outSocket.address+      & String    & 16             & -               & ''                  & -           & \verb+0.0.0.0+ & Address/interface to use for port\footnote{The default listens on all interfaces.}, \textit{e.g.} \verb+127.0.0.1+ \\
      \hline
      \verb+outSocket.expectedClients+      & UShort    & -             & -               & ''                  & Standard           & 0 & Number of clients required to be connected before \verb+run()+ method will proceed.\footnote{Probably useful only for testing and may incorrectly inhibit data flow.} \\
      \hline
      \verb+outSocket.port+         & UShort    & -              & -               & ''                  & 1025 - 65535 & -            & Output port to use if all data should remain combined \footnote{ICANN reserves up to 49151.} \footnote{Attempting to use a port that is used by another process will cause a fatal error.} \\
      \hline
      \verb+outSocket.ports+        & UShort    & -              & 256             & ''                  & -           & -             & A list of port numbers to listen on, with \verb+0+ indicating unused \footnote{See ``Performance and Resource Utilization.''} \footnote{This Property is only used when \texttt{port} is set to \texttt{0}.}\\
      \hline
      \verb+outSocket+ \verb+.messagesInStream+ & Bool & -       & 256             & ''                  & -           & false         & Write out data in ``message'' mode with embedded opcode \\
      \hline
      \verb+current+                & Struct    & -              & -               & Volatile           & -           & -             & Current statistics for each opcode \\
      \hline
      \verb+current.Total+          & Struct    & -              & -               & ''                  & -           & -             & Statistics across \textit{all} opcodes \\
      \hline
      \verb+current.Total.bytes+    & ULongLong & -              & -               & ''                  & Standard    & -             & Number of bytes received \\
      \hline
      \verb+current.Total.messages+ & ULongLong & -              & -               & ''                  & Standard    & -             & Number of messages received \\
      \hline
      \verb+current.Opcode+         & Struct    & -              & 256             & ''                  & -           & -             & Statistics for \textit{each} opcode \\
      \hline
      \verb+current.Opcode.*+       & Various   & -              & ''               & -                  & -           & -             & Various\footnote{Internal structure equivalent to \texttt{current.Total} and not explicitly shown.} \\
      \hline
      \verb+stopOn+                 & Struct    & -              & -               & Writable, Readable & -           & -             & Condition(s) required to have Worker report completion\footnote{Any matched condition will halt the processing.} \\
      \hline
      \verb+stopOn.Total+           & Struct    & -              & -               & ''                  & -           & -             & Stops if any non-zero value is exceeded when counting \textit{all} data received \\
      \hline
      \verb+stopOn.Total.bytes+     & ULongLong & -              & -               & ''                  & Standard    & 0             & Stop on number of bytes received \\
      \hline
      \verb+stopOn.Total.messages+  & ULongLong & -              & -               & ''                  & Standard    & 0             & Stop in number of messages received \\
      \hline
      \verb+stopOn.Opcode+          & Struct    & -              & 256             & ''                  & -           & -             & Stops if any non-zero value is exceeded when counting data received using a specific opcode \\
      \hline
      \verb+stopOn.Opcode.*+        & Various   & -              & -               & ''                  & -           & -             & Various\footnote{\label{stopon}Internal structure equivalent to \texttt{stopOn.Total} and not explicitly shown.} \\
      \hline
      \verb+stopOn.Any+             & Struct    & -              & -               & ''                  & -           & -             & Stops if any non-zero value is exceeded when counting data received using any single opcode \\
      \hline
      \verb+stopOn.Any.*+           & Various   & -              & -               & ''                  & -           & -             & Various\footnotemark[\getrefnumber{stopon}] \\
      \hline
      \verb+stopOn.ZLM+             & UShort    & -              & -               & ''                  & 0 - 256     & 0             & Stops if a \textbf{Z}ero \textbf{L}ength \textbf{M}essage is received using a given opcode.\footnote{Default is opcode 0; set to invalid opcode 256 if this feature is \textit{not} desired.} \\
      \hline
    \end{tabular}
  \end{scriptsize}
  \end{minipage}
  \section*{Worker Properties}
  \subsection*{\comp.rcc}
  Control Operations\footnote{All TCP connections are terminated in the \textbf{Stop} state, while listening ports are opened in the \textbf{Start} state. If the Component is \textbf{Stop}ped, any clients will be \emph{disconnected} (i.e.\ not paused) and must reconnect after it is \textbf{Start}ed.} Start, Stop

  \section*{Component Ports}
  \begin{scriptsize}
    \begin{tabular}{|M{2cm}|M{1.5cm}|M{4cm}|c|c|M{9cm}|}
      \hline
      \rowcolor{blue}
      Name & Producer & Protocol & Optional & Advanced            & Usage \\
      \hline
      in   & false    & -        & false    & numberofopcodes=256 & Data to be streamed to sockets(s) \\
      \hline
      out  & true     & -        & true     & numberofopcodes=256 & Data pass-through \\
      \hline
    \end{tabular}
  \end{scriptsize}

  \section*{Worker Interfaces}
  There are no implementation-specific interfaces for this component.
\end{landscape}

\section*{Performance and Resource Utilization}

\subsubsection*{\comp.rcc}
\begin{scriptsize}
  \begin{tabular}{|c|c|c|}
    \hline
    \rowcolor{blue}
    Processor Type                                & Processor Frequency & Run Function Time \\
    \hline
    linux-c6-x86\_64 Intel(R) Xeon(R) CPU E5-1607 & 3.00 GHz            & TBD               \\
    \hline
    linux-c7-x86\_64 Intel(R) Core(TM) i7-3630QM  & 2.40 GHz            & TBD               \\
    \hline
    linux-13\_3-arm ARMv7 Processor rev 0 (v7l)    & 666 MHz             & TBD               \\
    \hline
  \end{tabular}
\end{scriptsize}
\medskip

Each listening port requires system resources, such as an open file descriptor. When opening more than a handful of ports, the user may need to use \verb+ulimit+ to increase the number of open file descriptors. To do this temporarily, the command \verb+ulimit -n 2048+ can \textit{sometimes} fix the currently running shell. Consult the documentation for your Operating System to permanently increase the limit for all processes.
\medskip

TCP connections have a large overhead when compared to other transport processes, such as the OpenCPI internal messaging system. Currently, this component \textbf{does not} combine Messages to optimize the outbound connection, \textit{e.g.} taking into account TCP Maximum Segment Size (MSS). It is \textit{highly} recommended that users of this Component use a minimum message size of 4K or combine multiple messages in some way in an upstream Component.
\medskip

Data buffers are not returned to the framework nor sent out the optional ``out'' port until all TCP clients have received their data. This may result in throttling or a possible denial-of-service attack if a malicious client connects but never accepts data.

\section*{Test and Verification}

\subsubsection*{Usage (local/x86)}
After building the component, the user needs to type \verb+make tests RCC_CONTAINERS=1+ in the \textit{socket\_write.test} directory. Various properties and data flows will be tested to try to cover as many use cases as possible. \\

If the user would like to execute only one test, \verb+TESTS=test_XX+ can be added to the end of the command.

\subsubsection*{\textbf{Experimental}: Usage (remote/ARM)}
Full test environment configuration (\textit{e.g.} NFS mounting, \verb+OCPI_CDK_DIR+, etc.) on the remote GPP is beyond the scope of this document. The test procedures assume that both shells' current working directory is the \textit{socket\_write.test} directory (NFS-mounted on remote) and \verb+ocpirun+ is in the remote's current \verb+PATH+. NFS \textbf{must} be used for the scripts to properly verify the outputs. \\

In the host shell, the user types \verb+make tests IP=xx.xx.xx.xx+. A command that can be copied and then pasted into the remote shell will be displayed. \textbf{This command should be executed in less than a minute to ensure the test system begins listening before the host times out.} The timeout can be changed using the \verb+LISTEN_TIMEOUT+ variable. Once the remote shell returns to the bash prompt, pressing ``Enter'' on the host will begin the verification process. \\

Single tests can be performed in the same manner as documented above.

\paragraph{Specific Platform Note - Matchstiq-Z1}
~\\
\\
Some tests have had ``Segmentation Faults'' or ``Alignment Errors'' in certain scenarios on the Z1. The problem becomes most evident when there are multiple clients connected, but has been more rarely observed with even a single client. This seems to happen when both USB ports are used to simultaneously transmit a large amount of data, \textit{e.g.} high log-level output to a USB serial console as well as NFS-mounted output files over a USB-to-Ethernet adapter. The default test setup avoids triggering this by limiting output that is fed to the user, but users should be aware of this issue if non-default test scenarios are attempted. If \texttt{ssh} is used to have all data routed through the USB-to-Ethernet adapter, this failure mode is avoided.\par

\pagebreak

\subsubsection*{Detailed Theory of Operation}
\begin{flushleft}
  Each \verb+test_XX+ subdirectory has the following files:

  \begin{itemize}{}{}
    \item \texttt{description} - a one-line description of the test
    \item \texttt{application.xml} - the OAS XML for the test setup
    \item \texttt{portmap} - (optional) list of TCP ports paired to output files
    \item \texttt{localinclude.mk} - (optional) custom \verb+Makefile+ rules needed for test
    \item \texttt{golden.md5} - (optional) MD5 checksums of golden/expected output
    \item \texttt{generate.[sh|pl|py]} - (optional) script to generate test data
    \item \texttt{verify.sh} - (optional) script to verify output(s)
  \end{itemize}

  Data is sourced with a source component (often \verb+pattern+ or \verb+file_read+) within the OAS. If the former, the source data is encapsulated in the OAS. When the latter, a \verb+generate.py+ script generates the required data. Most OASs dump the ``\texttt{current}'' property to a file \verb+UUT.current.dump+, which is also confirmed to match expected output. Some tests connect a \verb+file_write_demux+ to the \verb+out+ port to verify pass-through operation.
  \medskip

  If \texttt{generate.sh} does not exist, a default one is created that will run \texttt{generate.pl} and/or \texttt{generate.py} if they exist and are executable. This default script is removed with \verb+make clean+.
  \medskip

  At test launch, if a file \texttt{portmap} exists, it launches a Python-based utility script \verb+busy_loop_socket.py+, which opens a client on a given port and repeatedly attempts to connect. Each line is a port number followed by a (relative) file path where the data is written upon successful connection.
  \medskip

  If \texttt{verify.sh} does not exist, a default one is created that will ensure the application did not time out and then run \texttt{md5sum} to verify all the checksums listed in \texttt{golden.md5}. This default script is also removed upon \verb+make clean+.
\end{flushleft}
\end{document}
