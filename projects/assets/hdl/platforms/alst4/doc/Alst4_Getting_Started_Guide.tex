\documentclass{article}
\iffalse
This file is protected by Copyright. Please refer to the COPYRIGHT file
distributed with this source distribution.

This file is part of OpenCPI <http://www.opencpi.org>

OpenCPI is free software: you can redistribute it and/or modify it under the
terms of the GNU Lesser General Public License as published by the Free Software
Foundation, either version 3 of the License, or (at your option) any later
version.

OpenCPI is distributed in the hope that it will be useful, but WITHOUT ANY
WARRANTY; without even the implied warranty of MERCHANTABILITY or FITNESS FOR A
PARTICULAR PURPOSE. See the GNU Lesser General Public License for more details.

You should have received a copy of the GNU Lesser General Public License along
with this program. If not, see <http://www.gnu.org/licenses/>.
\fi

% TODO: Version numbers?
\usepackage{graphicx}
\graphicspath{ {figures/} }
\usepackage{fancyhdr}
\usepackage{colortbl}
\usepackage[margin=.75in]{geometry}
\usepackage{hyperref}
\usepackage{listings}
\usepackage{xcolor}
\usepackage{textcomp}
\pagestyle{fancy}
\lhead{Board Support Package Documentation}
\rhead{ANGRYVIPER Team}
\renewcommand{\headrulewidth}{0pt}
\newcommand{\shellcmd}[1]{\texttt{\$ #1\\}}
\newcommand{\terminaloutput}[1]{\texttt{#1}}
\definecolor{blue}{rgb}{.5,1,1}
\definecolor{drkgreen}{rgb}{0,.6,0}
\begin{document}

\section*{Zipper Deprecation Notice:}
Beginning with OpenCPI Version 1.5, support for Lime Microsystems' Zipper card is now deprecated.

\section*{ALST4 Getting Started Guide}
\setcounter{section}{0}

\section{Hardware Prerequisites}
This section describes the hardware prerequisites required for an operational alst4 (Altera Stratix IV) platform using OpenCPI. The optional HSMC Debug Loopback and HSMC Debug Breakout Header cards are only intended for testing purposes. Also note that the slot configurations in Table \ref{table:supported_slots} are limited by what FPGA bitstreams are currently built by OpenCPI and not by what hardware configurations are theoretically possible using OpenCPI.\\ \\
Hardware prerequisites are as follows.
\begin{itemize}
\item A Stratix IV GX230 board, which has undergone an OpenCPI-specific initial one-time hardware setup \cite{alst4_hardware_setup} and is plugged into a PCIE slot of an x86 computer.
\item Optionally, one of the following HSMC card configurations in Table  \ref{table:supported_slots} may exist
\end{itemize}
\begin{center}
        \begin{table}[!htbp]
        \centering
        \caption{OpenCPI-supported Stratix IV hardware HSMC slot configurations}
        \label{table:supported_slots}
        \begin{tabular}{c|c|c|}
                \cline{2-3}
                & HSMC A slot & HSMC B slot \\ \hline
                \multicolumn{1}{|c|}{Test loopback A setup} & HSMC Debug Loopback Card & (empty)\\ \hline
                \multicolumn{1}{|c|}{Test loopback B setup} & (empty) & HSMC Debug Loopback Card \\ \hline
                \multicolumn{1}{|c|}{Test dual loopback setup} & HSMC Debug Loopback Card & HSMC Debug Loopback Card \\ \hline
                \multicolumn{1}{|c|}{Test breakout A setup} & HSMC Debug Breakout Header Card & (empty)\\ \hline
                \multicolumn{1}{|c|}{Test breakout B setup} & (empty) & HSMC Debug Breakout Header Card \\ \hline
                \multicolumn{1}{|c|}{Zipper A setup} & Modified\cite{zipper_mods} Zipper/MyriadRF & (empty)\\
                \multicolumn{1}{|c|}{ } & transceiver card & \\ \hline
                \multicolumn{1}{|c|}{Zipper B setup} & (empty) & Modified\cite{zipper_mods} Zipper/MyriadRF \\
                \multicolumn{1}{|c|}{ } & & transceiver card \\ \hline
        \end{tabular}
        \end{table}
\end{center}

\section{Software Prerequisites}
\begin{itemize}
\item A CentOS 6 or CentOS 7 operating system installed on the x86 computer.
\item Altera Quartus installed on the x86 computer. For more information refer to \cite{fpga_vendor_tool_guide}
\item OpenCPI framework and prerequisite RPMs installed on the x86 computer. For more information refer to \cite{rpm_installation_guide}
\item OpenCPI core project compiled for alst4.
\item OpenCPI assets project compiled for alst4.
\end{itemize}

\section{Driver Setup}
\begin{flushleft}
If you want to use more then 128KB of RAM, then you will need to reserve a
block of memory during the Linux kernel boot, using the memmap parameter.  The
memmap parameter takes a number of formats, but the one that is most useful to
us is the following: \\
\bigskip
	memmap=SIZE\$START \\
\bigskip
Where SIZE is the number of bytes to reserve in either hex or decimal, and
START is the physical address in hexidecimal bytes.  You *must* use even
page boundaries (0x1000 or 4096 bytes) for all addresses and sizes. \\
\subsection{Calculate Values in Preparation for Memory Reservation}
Start by running: \\
\lstset{language=bash, backgroundcolor=\color{lightgray}, columns=flexible, breaklines=true, prebreak=\textbackslash, basicstyle=\ttfamily, showstringspaces=false,upquote=true, aboveskip=\baselineskip, belowskip=\baselineskip}
\begin{lstlisting}
dmesg | grep BIOS
\end{lstlisting}
The output will look something like:
\begin{lstlisting}
BIOS-provided physical RAM map:
 BIOS-e820: 0000000000000000 - 000000000009f800 (usable)
 BIOS-e820: 000000000009f800 - 00000000000a0000 (reserved)
 BIOS-e820: 00000000000ca000 - 00000000000cc000 (reserved)
 BIOS-e820: 00000000000dc000 - 00000000000e4000 (reserved)
 BIOS-e820: 00000000000e8000 - 0000000000100000 (reserved)
 BIOS-e820: 0000000000100000 - 000000005fef0000 (usable)
 BIOS-e820: 000000005fef0000 - 000000005feff000 (ACPI data)
 BIOS-e820: 000000005feff000 - 000000005ff00000 (ACPI NVS)
 BIOS-e820: 000000005ff00000 - 0000000060000000 (usable)
 BIOS-e820: 00000000e0000000 - 00000000f0000000 (reserved)
 BIOS-e820: 00000000fec00000 - 00000000fec10000 (reserved)
 BIOS-e820: 00000000fee00000 - 00000000fee01000 (reserved)
 BIOS-e820: 00000000fffe0000 - 0000000100000000 (reserved)
\end{lstlisting}
 You want to select a (usable) section of memory and reserve a section of that
 memory.  Once the memory is reserved, the Linux kernel will ignore it.  In
 this example, there are 3 useable sections:\\
\begin{lstlisting}
 BIOS-e820: 0000000000000000 - 000000000009f800 (usable)
 BIOS-e820: 0000000000100000 - 000000005fef0000 (usable)
 BIOS-e820: 000000005ff00000 - 0000000060000000 (usable)
\end{lstlisting}
Due to the way Linux manages memory, it is recommended you pick a higher
address (above the first 24 bits).  The best choice is the second section
(pages 0x100-0x5fef0).  If you wanted to reserve 128MB, that would be
0x8000 pages.  Pick the end of the block (page 0x5fef0) and subtract the
number of pages, leaving 0x57ef0.  This would result in the following memmap
parameter:\\
\bigskip
memmap=128M\$0x57EF0000\\
\subsection{Configure Memory Reservation}
Once you've calculated your memmap parameter, you will need to add it to the
kernel command line in your boot loader. \\
\bigskip
For CentOS, you can use the utility ``grubby''. \\
\bigskip
This will add the parameter to all kernels in the startup menu. The single
quotes are REQUIRED or your shell will interpret the \$0: \\
\bigskip
For \textbf{\textit{CentOS6}}:\\
\begin{lstlisting}
sudo grubby --update-kernel=ALL --args=memmap='128M\$0x57EF0000'
\end{lstlisting}
\textbf{\textit{CentOS 7}} uses grub2, which requires a double backslash to not interpret it:\\
\begin{lstlisting}
sudo grubby --update-kernel=ALL --args=memmap='128M\\$0x57EF0000'
\end{lstlisting}

To verify the current kernel has the argument set:\\
\begin{lstlisting}
sudo -v
sudo grubby --info $(sudo grubby --default-kernel)
\end{lstlisting}

\textbf{\textit{CentOS 7}} users should see a SINGLE backslash before the \$, for example: \\
\begin{lstlisting}
args="ro rdblacklist=nouveau crashkernel=auto rd.lvm.lv=vg.0/root quiet audit=1 boot=UUID=96933cb5-f478-4933-a0d4-16953cf47f5c memmap=128M\$0x57EF0000 LANG=en_US.UTF-8"
\end{lstlisting}



If no longer desired, the parameter can also be removed:
\begin{lstlisting}
sudo grubby --update-kernel=ALL --remove-args=memmap
\end{lstlisting}

More information concerning grubby can be found at:\\
\url{https://access.redhat.com/documentation/en-US/Red_Hat_Enterprise_Linux/7/html/System_Administrators_Guide/sec-Making_Persistent_Changes_to_a_GRUB_2_Menu_Using_the_grubby_Tool.html}
\bigskip
... the memmap parameter:\\
\url{https://www.kernel.org/doc/html/latest/admin-guide/kernel-parameters.html}

\bigskip
Note: If you have other memmap parameters, e.g. for non-OpenCPI PCI cards,
then grubby usage will be different. The OpenCPI driver will use the first
memmap parameter on the command line OR the parameter ``opencpi\_memmap'' if it
is explicitly given. If this parameter is given, the standard memmap command
with the same parameters must ALSO be passed to the kernel.\\
\subsection{Apply Memory Reservation}
Reboot the system, making certain to boot from your new configuration.
\subsection{Verify Memory Reservation}
Once that's done, if you run 'dmesg' you should see something like this:\\
\bigskip
\begin{lstlisting}
dmesg | more
Linux version 2.6.18-128.el5 (mockbuild@hs20-bc1-7.build.redhat.com) (gcc version 4.1.2 20080704 (Red Hat 4.1.2-44)) #1 SMP Wed Dec 17 11:41:38 EST 2008
Command line: ro root=/dev/VolGroup00/LogVol00 rhgb quiet memmap=128M$0x57EF0000
BIOS-provided physical RAM map:
 BIOS-e820: 0000000000000000 - 000000000009f800 (usable)
 BIOS-e820: 000000000009f800 - 00000000000a0000 (reserved)
 BIOS-e820: 00000000000ca000 - 00000000000cc000 (reserved)
 BIOS-e820: 00000000000dc000 - 00000000000e4000 (reserved)
 BIOS-e820: 00000000000e8000 - 0000000000100000 (reserved)
 BIOS-e820: 0000000000100000 - 000000005fef0000 (usable)
 BIOS-e820: 000000005fef0000 - 000000005feff000 (ACPI data)
 BIOS-e820: 000000005feff000 - 000000005ff00000 (ACPI NVS)
 BIOS-e820: 000000005ff00000 - 0000000060000000 (usable)
 BIOS-e820: 00000000e0000000 - 00000000f0000000 (reserved)
 BIOS-e820: 00000000fec00000 - 00000000fec10000 (reserved)
 BIOS-e820: 00000000fee00000 - 00000000fee01000 (reserved)
 BIOS-e820: 00000000fffe0000 - 0000000100000000 (reserved)
user-defined physical RAM map:
 user: 0000000000000000 - 000000000009f800 (usable)
 user: 000000000009f800 - 00000000000a0000 (reserved)
 user: 00000000000ca000 - 00000000000cc000 (reserved)
 user: 00000000000dc000 - 00000000000e4000 (reserved)
 user: 00000000000e8000 - 0000000000100000 (reserved)
 user: 0000000000100000 - 0000000057ef0000 (usable)
 user: 0000000057ef0000 - 000000005fef0000 (reserved)  <== New
 user: 000000005fef0000 - 000000005feff000 (ACPI data)
 user: 000000005feff000 - 000000005ff00000 (ACPI NVS)
 user: 000000005ff00000 - 0000000060000000 (usable)
 user: 00000000e0000000 - 00000000f0000000 (reserved)
 user: 00000000fec00000 - 00000000fec10000 (reserved)
 user: 00000000fee00000 - 00000000fee01000 (reserved)
 user: 00000000fffe0000 - 0000000100000000 (reserved)
DMI present.
\end{lstlisting}

You will see a new (reserved) area between the second (useable) section and the
(ACPI data) section.\\
\bigskip
Now, when you run the 'make load' script, it will detect the new reserved
area, and pass that data to the opencpi kernel module. \\
\end{flushleft}


\section{Driver Notes}
\iffalse
This file is protected by Copyright. Please refer to the COPYRIGHT file
distributed with this source distribution.

This file is part of OpenCPI <http://www.opencpi.org>

OpenCPI is free software: you can redistribute it and/or modify it under the
terms of the GNU Lesser General Public License as published by the Free Software
Foundation, either version 3 of the License, or (at your option) any later
version.

OpenCPI is distributed in the hope that it will be useful, but WITHOUT ANY
WARRANTY; without even the implied warranty of MERCHANTABILITY or FITNESS FOR A
PARTICULAR PURPOSE. See the GNU Lesser General Public License for more details.

You should have received a copy of the GNU Lesser General Public License along
with this program. If not, see <http://www.gnu.org/licenses/>.
\fi

% This is for inserting into various "Getting Started" Guides
% First, turn off indenting to avoid all the flushleft
\newlength{\savedparindentdrvr}%
\setlength{\savedparindentdrvr}{\parindent}%
\setlength{\parindent}{0pt} % Don't indent all paragraphs
\providecommand{\forceindent}{\leavevmode{\parindent=1em\indent}}%

When available, the driver will attempt to make use of the CMA region for direct memory access. In use cases where many memory allocations are made, the user may receive the following kernel message:

\lstset{language=bash, backgroundcolor=\color{lightgray}, columns=flexible, breaklines=true, prebreak=\textbackslash, basicstyle=\ttfamily, showstringspaces=false,upquote=true, aboveskip=\baselineskip, belowskip=\baselineskip}
\begin{lstlisting}
alloc_contig_range test_pages_isolated([memory start], [memory end]) failed
\end{lstlisting}

This is a kernel warning, but does not indicate that a memory allocation failure occurred, only that the CMA engine could not allocate memory in the first pass. Its default behavior is to make a second pass and if that succeeded the end user should not see any more error messages. An actual allocation failure will generate unambiguous error messages.

\setlength{\parindent}{\savedparindentdrvr}%


\section{Loading the OpenCPI driver}
When OpenCPI is installed via RPMs, the OpenCPI driver should have been installed. However, when developing with source OpenCPI, the user is required to manage the loading of the OpenCPI driver. \\
The following terminal outputs are intended to provide the user with expected behavior of when the driver is not and is loaded. The user should note that only when the driver is installed can the alst4 be discovered as a valid OpenCPI container.

\begin{lstlisting}
ocpidriver unload
The driver module was successfully unloaded.

ocpidriver load
Found generic reserved DMA memory on the linux boot command line and assuming it is for OpenCPI: [memmap=128M$0x1000000]
Driver loaded successfully.

ocpidriver unload
The driver module was successfully unloaded.

ocpirun -C
OCPI( 2:816.0497): When searching for PCI device '0000:03:00.0': Can't open /dev/mem, forgot to load the driver? sudo?
OCPI( 2:816.0499): When searching for PCI device '0000:08:00.0': Can't open /dev/mem, forgot to load the driver? sudo?
OCPI( 2:816.0544): In HDL Container driver, got PCI search error: Can't open /dev/mem, forgot to load the driver? sudo?
Available containers:
 #  Model Platform       OS     OS-Version  Arch     Name
 0  rcc   centos7        linux  c7          x86_64   rcc0
 
ocpidriver load
Found generic reserved DMA memory on the linux boot command line and assuming it is for OpenCPI: [memmap=128M$0x1000000]
Driver loaded successfully.

ocpirun -C
Available containers:
 #  Model Platform       OS     OS-Version  Arch     Name
 0  hdl   ml605                                      PCI:0000:08:00.0
 1  hdl   alst4                                      PCI:0000:03:00.0
 2  rcc   centos7        linux  c7          x86_64   rcc0
\end{lstlisting}

\section{Proof of Operation}
The following commands may be run in order to verify correct OpenCPI operation on the x86/Stratix IV system.\\ \\
Existence of alst4 RCC/HDL containers may be verified by running the following command and verifying that similar output is produced.\\
\lstset{language=bash, backgroundcolor=\color{lightgray}, columns=flexible, breaklines=true, prebreak=\textbackslash, basicstyle=\ttfamily, showstringspaces=false,upquote=true, aboveskip=\baselineskip, belowskip=\baselineskip}
\begin{lstlisting}
ocpirun -C
Available containers:
 #  Model Platform       OS     OS-Version  Arch     Name
 0  rcc   centos7        linux  c7          x86_64   rcc0
 1  hdl   alst4                                      PCI:0000:02:00.0
\end{lstlisting}
Operation of the RCC container can be verified by running the hello application via the following command and verifying that identical output is produced. Note that the OCPI\_LIBRARY\_PATH environment variable must be setup to include the hello\_world.rcc built shared object file prior to running this command.
\begin{lstlisting}
ocpirun -t 1 assets/applications/hello.xml
Hello, world
\end{lstlisting}
Simultaneous RCC/HDL container operation can be verified by running the testbias application via the following command and verifying that identical output is produced. Note that the OCPI\_LIBRARY\_PATH environment variable must be setup correctly for your system prior to running this command.\\
\begin{lstlisting}
ocpirun -d -m bias=hdl assets/applications/testbias.xml
Property  0: file_read.fileName = "test.input" (cached)
Property  1: file_read.messagesInFile = "false" (cached)
Property  2: file_read.opcode = "0" (cached)
Property  3: file_read.messageSize = "16"
Property  4: file_read.granularity = "4" (cached)
Property  5: file_read.repeat = "<unreadable>"
Property  6: file_read.bytesRead = "0"
Property  7: file_read.messagesWritten = "0"
Property  8: file_read.suppressEOF = "false"
Property  9: file_read.badMessage = "false"
Property 10: file_read.ocpi_debug = "false" (parameter)
Property 11: file_read.ocpi_endian = "little" (parameter)
Property 12: bias.biasValue = "16909060" (cached)
Property 13: bias.ocpi_debug = "false" (parameter)
Property 14: bias.ocpi_endian = "little" (parameter)
Property 15: bias.test64 = "0"
Property 16: file_write.fileName = "test.output" (cached)
Property 17: file_write.messagesInFile = "false" (cached)
Property 18: file_write.bytesWritten = "0"
Property 19: file_write.messagesWritten = "0"
Property 20: file_write.stopOnEOF = "true" (cached)
Property 21: file_write.ocpi_debug = "false" (parameter)
Property 22: file_write.ocpi_endian = "little" (parameter)
Property  3: file_read.messageSize = "16"
Property  5: file_read.repeat = "<unreadable>"
Property  6: file_read.bytesRead = "4000"
Property  7: file_read.messagesWritten = "251"
Property  8: file_read.suppressEOF = "false"
Property  9: file_read.badMessage = "false"
Property 15: bias.test64 = "0"
Property 18: file_write.bytesWritten = "4000"
Property 19: file_write.messagesWritten = "250"
\end{lstlisting}

\subsection*{Known Issues}
\subsubsection*{JTAG Daemon}
When loading FPGA bitstreams onto the alst4 FPGA (which can occur when running either \terminaloutput{ocpihdl load} or \terminaloutput{ocpirun}), multiple issues exists with the Altera jtag daemon which may cause the FPGA loading to fail. The following is an example of the terminal output when this failure occurs:
\begin{lstlisting}
Checking existing loaded bitstream on OpenCPI HDL device "PCI:0000:0b:00.0"...
OpenCPI FPGA at PCI 0000:0b:00.0: bitstream date Wed Oct 19 16:04:45 2016, platf
orm "alst4", part "ep4sgx230k", UUID 482195b4-9637-11e6-8002-d76b7b3cbb11
Existing loaded bitstream looks ok, proceeding to snapshot the PCI configuration
(into /tmp/ocpibitstream15980.1).
Scanning for JTAG cables...
Found cable "USB-Blaster [3-11]" to use for device "PCI:0000:0b:00.0" (no serial
number specified).
Error: did not find part ep4sgx230k in the jtag chain for cable USB-Blaster [3-1
1].
Look at /tmp/ocpibitstream15980.log for details.
Error: Could not find jtag position for part ep4sgx230k on JTAG cable "USB-Blast
er [3-11]".
OpenCPI FPGA at PCI 0000:0b:00.0: bitstream date Wed Oct 19 16:04:45 2016, platf
Exception thrown: Bitstream loading error (exit code: 1) loading "../../hdl/ass
emblies/dc offset iq imbalance mixer cic dec timestamper/container-dc offset iq imba
lance mixer cic dec timestamper alst4 base alst4 adc hsmc port b/target-stratix4/dc
offset iq imbalance mixer cic dec timestamper alst4 base alst4 adc hsmc port b.sof.gz
" on HDL device "PCI:0000:0b:00.0" with command: /opt/opencpi/cdk//scripts/loadBi
tStream "../../hdl/assemblies/dc offset iq imbalance mixer cic dec timestamper/conta
iner-dc offset iq imbalance mixer cic dec timestamper alst4 base alst4 adc hsmc port b
/target-stratix4/dc offset iq imbalance mixer cic dec timestamper alst4 base alst4 adc
hsmc port b.sof.gz" "PCI:0000:0b:00.0" "alst4" "ep4sgx230k" "" ""
\end{lstlisting}
The failure may also manifest as a permissions issue:
\begin{lstlisting}
Scanning for JTAG cables...
JTAG cable setup for platform "alst4" failed.
Dump of /tmp/ocpibitstream5904.cables:
********************************************************************************
Cable "USB-Blaster variant [3-7]": cannot get serial number.
********************************************************************************
Dump of /tmp/ocpibitstream5904.log:
********************************************************************************
Error when locking chain - Insufficient port permissions
********************************************************************************
\end{lstlisting}

\noindent The follow commands implement a known remedy for each of the aforementioned errors:
\begin{lstlisting}
sudo killall jtagd
sudo chmod 755 /sys/kernel/debug/usb/devices
sudo chmod 755 /sys/kernel/debug/usb
sudo chmod 755 /sys/kernel/debug
sudo mount --bind /dev/bus /proc/bus
sudo ln -s /sys/kernel/debug/usb/devices /proc/bus/usb/devices
sudo <quartus_directory>/bin/jtagd
sudo <quartus_directory>/bin/jtagconfig
\end{lstlisting}

\subsubsection*{Single Port of Data from CPU to FPGA} % AV-3783
\label{bug:3783}
The current implementations of the PCI-e Specification on the this platform only correctly implements data flow from the CPU to the FPGA under certain configurations which must be met when defining new Assemblies:
\begin{itemize}
\item At most a single data port with CPU-to-FPGA data flow. Port connection must also be one of:
\begin{enumerate}
\item defined in a single-worker Assembly XML using the worker ``\texttt{Externals='true'}'' attribute/value and the DefaultContainer used (DefaultContainer not defined in Assembly Makefile), or
\item the first External Assembly Connection defined in the Assembly XML and the DefaultContainer used, or
\item the first Interconnect Container Connection defined in a Container XML (Default Container must be disabled via ``\texttt{DefaultContainer=}'' in the Assembly Makefile).
\end{enumerate}
\end{itemize}
Note that this applies to the TX/DAC data path connections for bitstreams with transceiver transmit data flow from a CPU (e.g. RCC worker to FPGA TX/DAC data path). See \path{projects/assets/hdl/assemblies/empty/cnt_1rx_1tx_bypassasm_fmcomms_2_3_lpc_LVDS_ml605.xml} as an example.

\pagebreak
  \begin{thebibliography}{1}


  \bibitem{alst4_hardware_setup} ALST4 Hardware Setup\\
	 \url{https://opencpi.github.io/assets/alst4_hardware_setup.pdf}
  \bibitem{fpga_vendor_tool_guide} FPGA Vendor Tools Guide\\
	 \url{https://opencpi.github.io/FPGA_Vendor_Tools_Installation_Guide.pdf}
	   \bibitem{rpm_installation_guide} OpenCPI RPM Installation Guide\\
	 \url{https://opencpi.github.io/RPM_Installation_Guide.pdf}
	   \bibitem{zipper_mods} Required Modifications for Myriad-RF 1 and Zipper Daughtercards\\
	 \url{https://opencpi.github.io/assets/Required_Modifications_for_Myriad-RF_1_Zipper_Daughtercards.pdf}

  \end{thebibliography}

\end{document}
