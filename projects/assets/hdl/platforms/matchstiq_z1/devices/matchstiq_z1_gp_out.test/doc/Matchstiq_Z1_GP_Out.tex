\documentclass{article}
\iffalse
This file is protected by Copyright. Please refer to the COPYRIGHT file
distributed with this source distribution.

This file is part of OpenCPI <http://www.opencpi.org>

OpenCPI is free software: you can redistribute it and/or modify it under the
terms of the GNU Lesser General Public License as published by the Free Software
Foundation, either version 3 of the License, or (at your option) any later
version.

OpenCPI is distributed in the hope that it will be useful, but WITHOUT ANY
WARRANTY; without even the implied warranty of MERCHANTABILITY or FITNESS FOR A
PARTICULAR PURPOSE. See the GNU Lesser General Public License for more details.

You should have received a copy of the GNU Lesser General Public License along
with this program. If not, see <http://www.gnu.org/licenses/>.
\fi

\author{} % Force author to be blank
%----------------------------------------------------------------------------------------
% Paper size, orientation and margins
%----------------------------------------------------------------------------------------
\usepackage{geometry}
\geometry{
	letterpaper,			% paper type
	portrait,				% text direction
	left=.75in,				% left margin
	top=.75in,				% top margin
	right=.75in,			% right margin
	bottom=.75in			% bottom margin
 }
%----------------------------------------------------------------------------------------
% Header/Footer
%----------------------------------------------------------------------------------------
\usepackage{fancyhdr} \pagestyle{fancy} % required for fancy headers
\renewcommand{\headrulewidth}{0.5pt}
\renewcommand{\footrulewidth}{0.5pt}
\rhead{\small{ANGRYVIPER Team}}
%----------------------------------------------------------------------------------------
% Appendix packages
%----------------------------------------------------------------------------------------
\usepackage[toc,page]{appendix}
%----------------------------------------------------------------------------------------
% Defined Commands & Renamed Commands
%----------------------------------------------------------------------------------------
\renewcommand{\contentsname}{Table of Contents}
\renewcommand{\listfigurename}{List of Figures}
\renewcommand{\listtablename}{List of Tables}
\newcommand{\todo}[1]{\textcolor{red}{TODO: #1}\PackageWarning{TODO:}{#1}} % To do notes
\newcommand{\code}[1]{\texttt{#1}} % For inline code snippet or command line
%----------------------------------------------------------------------------------------
% Various packages
%----------------------------------------------------------------------------------------
\usepackage{hyperref} % for linking urls and lists
\usepackage{graphicx} % for including pictures by file
\usepackage{listings} % for coding language styles
\usepackage{rotating} % for sideways table
\usepackage{pifont}   % for sideways table
\usepackage{pdflscape} % for landscape view
%----------------------------------------------------------------------------------------
% Table packages
%----------------------------------------------------------------------------------------
\usepackage{longtable} % for long possibly multi-page tables
\usepackage{tabularx} % c=center,l=left,r=right,X=fill
\usepackage{float}
\floatstyle{plaintop}
\usepackage[tableposition=top]{caption}
\newcolumntype{P}[1]{>{\centering\arraybackslash}p{#1}}
\newcolumntype{M}[1]{>{\centering\arraybackslash}m{#1}}
%----------------------------------------------------------------------------------------
% Block Diagram / FSM Drawings
%----------------------------------------------------------------------------------------
\usepackage{tikz}
\usetikzlibrary{shapes,arrows,fit,positioning}
\usetikzlibrary{automata} % used for the fsm
\usetikzlibrary{calc} % For duplicating clients
\usepgfmodule{oo} % To define a client box
%----------------------------------------------------------------------------------------
% Colors Used
%----------------------------------------------------------------------------------------
\usepackage{colortbl}
\definecolor{blue}{rgb}{.7,.8,.9}
\definecolor{ceruleanblue}{rgb}{0.16, 0.32, 0.75}
\definecolor{drkgreen}{rgb}{0,0.6,0}
\definecolor{deepmagenta}{rgb}{0.8, 0.0, 0.8}
\definecolor{cyan}{rgb}{0.0,0.6,0.6}
\definecolor{maroon}{rgb}{0.5,0,0}
\usepackage{multirow}
%----------------------------------------------------------------------------------------
% Update the docTitle and docVersion per document
%----------------------------------------------------------------------------------------
\def\docTitle{Component Data Sheet}
\def\docVersion{1.5}
%----------------------------------------------------------------------------------------
\date{Version \docVersion} % Force date to be blank and override date with version
\title{\docTitle}
\lhead{\small{\docTitle}}

\graphicspath{ {figures/} }

\def\comp{Matchstiq\_Z1\_GP\_Out}
\edef\ecomp{matchstiq_z1_gp_out}


\begin{document}

\section*{Summary - Matchstiq\_Z1\_GP\_Out}
	\begin{tabular}{|c|M{13.5cm}|}
		\hline
		\rowcolor{blue}
		• & • \\
		\hline
		Name & matchstiq\_z1\_gp\_out \\
		\hline
		Version & v1.5 \\
		\hline
		Release Date & 5/2019 \\
		\hline
		Component Library & ocpi.assets.platforms.matchstiq\_z1.devices \\
		\hline
		Workers & matchstiq\_z1\_gp\_out.hdl \\
		\hline
		Tested Platforms & xsim, matchstiq\_z1 \\
		\hline
	\end{tabular}
	
\begin{center}
	\textit{\textbf{Revision History}}
		\begin{table}[H]
		\label{table:revisions} % Add "[H]" to force placement of table
			\begin{tabularx}{\textwidth}{|c|X|l|}
			\hline
			\rowcolor{blue}
			\textbf{Revision} & \textbf{Description of Change} & \textbf{Date} \\
		    \hline
		    v1.5 & Initial Release & 5/2019 \\
		    \hline
			\end{tabularx}
		\end{table}
	\end{center}	
	
\section*{Functionality}
\begin{flushleft}
The Matchstiq-Z1 GP Out device worker controls the three GPIO pins, FPGA\_GPIO1, FPGA\_GPIO2, and FPGA\_GPIO3 present on the Matchstiq-Z1 platform. It is configured for general purpose output.

\end{flushleft}

\section*{Worker Implementation Details}
\begin{flushleft}

The Matchstiq-Z1 GP Out device worker can control the GPIO pins via the property \texttt{mask\_data}, it's (optional) input port, and the devsignal, \texttt{dev\_gp}. The devsignal only controls FPGA\_GPIO1, while the property and the (optional) port control all 3 pins. \newline

The (optional) input port uses a protocol that has one opcode and contains data and a mask. See \texttt{gpio\_protocol.xml} in \path{ocpi.core/specs/gpio_protocol.xml}  for further details on the protocol. \newline

The devsignal \texttt{dev\_gp} is controlled by the lime\_tx device worker through the lime\_tx's \texttt{dev\_txen\_dac\_in.txen} devsignal. This provides a way to control a device when transmit is enabled. By default FPGA\_GPIO1 is controlled by lime\_tx. When the Matchstiq-Z1 GP Out device worker is in the reset state, this signal is still controlled by lime\_tx. \newline

All three ways of controlling the GPIO pins set the values of pins using a data and mask. In order for a GPIO pin to take on the value of a set data bit, the corresponding mask bit has to be set. The MSW of \texttt{mask\_data} and the data port data, must be the mask and LSW must be the data and the 3 LSB of the mask and data correspond to the three GPIO pins of the Matchstiq-Z1. \newline

The \texttt{input\_mask} property provides a way for enabling or disabling the use of the property \texttt{"mask\_data"}; in port "data" and "mask"; or the devsignal "data" and "mask", if it desired to enable or disable any one of these ways of controlling the GPIO pins. By default all three are enabled.

\end{flushleft}


\section*{Block Diagrams}
	\subsection*{Top level}
	\makeatletter
\newcommand{\gettikzxy}[3]{%
  \tikz@scan@one@point\pgfutil@firstofone#1\relax
  \edef#2{\the\pgf@x}%
  \edef#3{\the\pgf@y}%
}
\makeatother
\pgfooclass{clientbox}{ % This is the class clientbox
    \method clientbox() { % The clientbox
    }
    \method apply(#1,#2,#3,#4) { % Causes the clientbox to be shown at coordinate (#1,#2) and named #3
        \node[rectangle,draw=white,fill=white] at (#1,#2) (#3) {#4};
    }
}
\pgfoonew \myclient=new clientbox()
\begin{center}
\begin{tikzpicture}[% List of styles applied to all, to override specify on a case-by-case
					every node/.style={
						align=center,  		% use this so that the "\\" for line break works
						minimum size=1.5cm	% creates space above and below text in rectangle
						},
					every edge/.style={draw,thick}
					]
\node[rectangle,ultra thick,draw=black,fill=blue,minimum width=10 cm](R1){ Parameter Properties: \\ simulation\_p \\ \\ matchstiq\_z1\_gp\_out \\};
		\node[rectangle,draw=white,fill=white](R2)[above= of R1]{Non-parameter Properties:\\ See property table};
		\path[->]
		(R2)edge []	node [] {} (R1)
		(R1)edge []	node [] {} (R2)
		;					
		
    \gettikzxy{(R1)}{\rx}{\ry}
    \myclient.apply(\rx - 270,\ry + 20,C1, ``in'' StreamInterface \\  \texttt{(}connection optional\texttt{)} \\ mask and data \texttt{(}MSW: mask, LSW: data\texttt{)});
    \path[<-]($(R1.west) + (-0 pt,+20 pt)$) edge [] node [] {} (C1);
    \myclient.apply(\rx - 120,\ry-80,C1, ``dev\_gp'' \\ devsignal port \\ \texttt{(}see worker interfaces\texttt{)} );
    \path[<-]($(R1.south) + (-120 pt,0)$) edge [] node [] {} (C1);
    \myclient.apply(\rx + 0,\ry-80,C1, ``dev\_gp\_em'' \\ devsignal port \\ \texttt{(}see worker interfaces\texttt{)} );
    \path[->]($(R1.south) + (0 pt,0)$) edge [] node [] {} (C1);
\myclient.apply(\rx + 120,\ry-80,C1, Signals \\ see signals table);
    \path[->]($(R1.south) + (120 pt,0)$) edge [] node [] {} (C1);

\end{tikzpicture}
\end{center}

\section*{Source Dependencies}
\subsection*{matchstiq\_z1\_gp\_out.hdl}
	\begin{itemize}
		\item assets/hdl/platforms/matchstiq\_z1/devices/matchstiq\_z1\_gp\_out.hdl/matchstiq\_z1\_gp\_out.vhd
	\end{itemize}

\subsection*{matchstiq\_z1\_gp\_out\_em.hdl}
	\begin{itemize}
		\item assets/hdl/platforms/matchstiq\_z1/devices/matchstiq\_z1\_gp\_out\_em.hdl/matchstiq\_z1\_gp\_out\_em.vhd
\item assets/hdl/primitives/misc\_prims/misc\_prims\_pkg.vhd
	      \subitem assets/hdl/primitives/misc\_prims/edge\_detector/src/edge\_detector.vhd
\end{itemize}

\begin{landscape}
\section*{Worker Properties}

\begin{flushleft}

        \begin{scriptsize}
        \begin{tabular}{|p{2.2cm}|p{1cm}|p{1cm}|p{2cm}|p{1.5cm}|p{2cm}|p{1.5cm}|p{1.5cm}|p{6cm}|}
                \hline
                \rowcolor{blue}
                Name & Type & Default & SequenceLength & ArrayLength & ArrayDimensions & Parameter  & Accessibility & Usage \\
                \hline
                \verb+input_mask+ & uChar & 0x07 & - & - & - & false & Writable & Bitfield that allows enabling or disabling the use of the property "mask\_data"; in port "data" and "mask"; or the devsignal "data" and "mask". Bit 0 is the property, bit 1 is the in port, and bit 2 is the devsignal. If a bit is a 1 then the corresponding way of controlling the GPIO pin is enabled. \\
                \hline
                \verb+mask_data+ & uLong & 0 & - & - & - & false & Volatile, Writable & Bitfield containing the data to write the GPIO pins and the mask. The mask allows setting GPIO pins on or off in a single operation. The MSW must be the mask and LSW must be the data and the 3 LSB of the mask and data correspond to the 3 GPIO pins of the Matchstiq-Z1. For example if mask\_data = 0x00010003, the mask = 0x0001 and data = 0x0003. This would set gpio1 to 1. \\
                \hline
                \verb+simulation_p+ & bool & false & - & - & - & true & - & If true generate circuits for simulation logic. \\
                \hline
        \end{tabular}
        \end{scriptsize}
 

\end{flushleft}

 
\section*{Component Ports}

        \begin{scriptsize}
                \begin{tabular}{|M{2.5cm}|M{2cm}|M{2cm}|M{3cm}|M{11cm}|}
                        \hline
                        \rowcolor{blue}
                        Name & Protocol & Producer & Optional & Usage\\
                        \hline
                        in
                        & gpio\_protocol
                        & false
                        & true
                        & Data to write GPIO pins \\
                        \hline
                \end{tabular}
			\end{scriptsize}
			
\section*{Worker Interfaces}
\subsection*{matchstiq\_z1\_gp\_out.hdl}
\begin{scriptsize}
\begin{tabular}{|M{5cm}|M{4cm}|M{4cm}|c|M{6.5cm}|M{6cm}|}
            \hline
            \rowcolor{blue}
            Type    & Name & DataWidth (b) & Advanced  & Usage     \\
            \hline
            StreamInterface & in   & 32  & - & Data to write GPIO pins \\
           \hline

\end{tabular}
\end{scriptsize} \\ \\

\begin{scriptsize}
		\hskip-0.5cm \begin{tabular}{|p{3cm}|p{2cm}|p{1cm}|p{1.25cm}|p{1.25cm}|p{1cm}|p{1.4cm}|p{0.9cm}|p{7.2cm}|}
			\hline
			\rowcolor{blue}
			Type                       & Name                            & Count & Optional & Master                & Signal                & Direction                  & Width                    & Description                                                                                                                  \\
			\hline
			\multirow{4}{*}{DevSignal{}} & \multirow{4}{*}{dev\_gp} & \multirow{4}{*}{1} & \multirow{4}{*}{True} & \multirow{4}{*}{False}  & data & Output & 1 & Controlled by dev\_txen\_dac\_in.txen within the lime\_tx device worker. Used to control GPIO1 pin. \\
			\cline{6-9}
			&             &        &     &      & mask     & Output     & 1      & Controlled by dev\_txen\_dac\_in.txen within the lime\_tx device worker. Used in logic for controlling FPGA\_GPIO1 pin. \\
			\cline{6-9}
			\hline
			DevSignal     & dev\_gp\_em   & 1  & False & True & enable & Output & 1      & Controls when the matchstiq\_z1\_gp\_em device emulator should start sending messages. \\
			\hline
		\end{tabular}
	\end{scriptsize}
	
\section*{Signals}
	\begin{scriptsize}
		\begin{tabular}{|M{2.5cm}|M{3cm}|M{6.5cm}|M{9.2cm}|}
			\hline
			\rowcolor{blue}
			Name         & Type   & Width (b) & Description                       \\
			\hline
			gpio1    & out & 1 & Output to GPIO pins            \\
			\hline
			gpio2    & out & 1 & Output to GPIO pins            \\
			\hline
			gpio3    & out & 1 & Output to GPIO pins            \\
			\hline
		\end{tabular}
	\end{scriptsize}
\end{landscape}

\section*{Control Timing and Signals}
\begin{flushleft}
The Matchstiq-Z1 GP Out device worker uses the clock from the Control Plane and standard Control Plane signals.
\end{flushleft}

\begin{landscape}
\section*{Worker Configuration Parameters}
\subsubsection*{\comp.hdl}
\input{../../\ecomp.hdl/configurations.inc}
\section*{Performance and Resource Utilization}
\subsubsection*{\comp.hdl}
\input{../../\ecomp.hdl/utilization.inc}
\end{landscape}



\section*{Test and Verification}
\normalsize

\begin{flushleft}

The matchstiq\_z1\_gp\_out device worker unit has three test cases and utilizes the matchstiq\_z1\_gp\_out\_em device emulator. Case 1 tests controlling the GPIO pins via the \texttt{mask\_data} property, case 2 tests controlling them via the input port, and case 3 tests controlling via the \texttt{dev\_gp} devsignal. \newline

For the mask\_data property and the data port, the tests that are done are: setting data bits high but not setting masks, setting data bits high and setting the appropriate mask bits high, clear data by setting data to 0x0000 and mask to 0x0007,
and then set data bits high but only set some of the appropriate masks high.\newline

For the devsignal the mask is held high but the data is toggled on and off. \newline
  
The \path{generate.py} script generates the input data and generates golden data files. For case 1 and 3 the input file is a 0 byte file since the input port data is ignored in these two cases. A \path{.golden.data} file is generated for each case. \newline

The \path{verify.py} script checks that the output data matches the expected output data contained in the \path{.golden.data} files.

\end{flushleft}
\end{document}
