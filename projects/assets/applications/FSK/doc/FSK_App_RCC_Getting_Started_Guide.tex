\iffalse
This file is protected by Copyright. Please refer to the COPYRIGHT file
distributed with this source distribution.

This file is part of OpenCPI <http://www.opencpi.org>

OpenCPI is free software: you can redistribute it and/or modify it under the
terms of the GNU Lesser General Public License as published by the Free Software
Foundation, either version 3 of the License, or (at your option) any later
version.

OpenCPI is distributed in the hope that it will be useful, but WITHOUT ANY
WARRANTY; without even the implied warranty of MERCHANTABILITY or FITNESS FOR A
PARTICULAR PURPOSE. See the GNU Lesser General Public License for more details.

You should have received a copy of the GNU Lesser General Public License along
with this program. If not, see <http://www.gnu.org/licenses/>.
\fi

%----------------------------------------------------------------------------------------
% Update the docTitle and docVersion per document
%----------------------------------------------------------------------------------------
\def\docTitle{OpenCPI\\ FSK App RCC Getting Started Guide}
\def\docVersion{1.6}
%----------------------------------------------------------------------------------------
\def\snippetpath{../../../../../doc/av/tex/snippets}
% Usage:
% \def\snippetpath{../../../../../doc/av/tex/snippets/}
% % Usage:
% \def\snippetpath{../../../../../doc/av/tex/snippets/}
% % Usage:
% \def\snippetpath{../../../../../doc/av/tex/snippets/}
% \input{\snippetpath/includes}
% From then on, you can use "input" With no paths to get to "snippets"
% You also get all "major" snippets not part of the global LaTeX_Header
% NOTE: If not using the global LaTeX_Header, you need to
% \usepackage{ifthen} to use the \githubio macro

\hyphenation{ANGRY-VIPER} % Tell it where to hyphenate
\hyphenation{Cent-OS} % Tell it where to hyphenate
\hyphenation{install-ation} % Tell it where to hyphenate

\newcommand{\todo}[1]{\textcolor{red}{TODO: #1}\PackageWarning{TODO:}{#1}} % To do notes
\newcommand{\code}[1]{\texttt{#1}} % For inline code snippet or command line
\newcommand{\sref}[1]{Section~\ref{#1}} % To quickly reference a section

% To quickly reference a versioned PDF on github.io
% \def\ocpiversion{1.5.0}
\def\ocpiversion{1.5.0rc4} % TEMPORARY

% This gives a link to github.io document. By default, it puts the filename.
% You can optionally change the link, e.g.
% \githubio{FPGA\_Vendor\_Tools\_Installation\_Guide.pdf} vs.
% \githubio[\textit{FPGA Vendor Tools Installation Guide}]{FPGA\_Vendor\_Tools\_Installation\_Guide.pdf}
% or if you want the raw ugly URL to come out, \githubioURL{FPGA_Vendor_Tools_Installation_Guide.pdf}
\newcommand{\githubio}[2][]{% The default is for FIRST param!
\href{http://opencpi.github.io/releases/\ocpiversion/#2}{\ifthenelse{\equal{#1}{}}{\texttt{#2}}{#1}}}
\newcommand{\githubioURL}[1]{\url{http://opencpi.github.io/releases/\ocpiversion/#1}}

% Fix import paths
\makeatletter
\def\input@path{{\snippetpath/}}
\makeatother

% From then on, you can use "input" With no paths to get to "snippets"
% You also get all "major" snippets not part of the global LaTeX_Header
% NOTE: If not using the global LaTeX_Header, you need to
% \usepackage{ifthen} to use the \githubio macro

\hyphenation{ANGRY-VIPER} % Tell it where to hyphenate
\hyphenation{Cent-OS} % Tell it where to hyphenate
\hyphenation{install-ation} % Tell it where to hyphenate

\newcommand{\todo}[1]{\textcolor{red}{TODO: #1}\PackageWarning{TODO:}{#1}} % To do notes
\newcommand{\code}[1]{\texttt{#1}} % For inline code snippet or command line
\newcommand{\sref}[1]{Section~\ref{#1}} % To quickly reference a section

% To quickly reference a versioned PDF on github.io
% \def\ocpiversion{1.5.0}
\def\ocpiversion{1.5.0rc4} % TEMPORARY

% This gives a link to github.io document. By default, it puts the filename.
% You can optionally change the link, e.g.
% \githubio{FPGA\_Vendor\_Tools\_Installation\_Guide.pdf} vs.
% \githubio[\textit{FPGA Vendor Tools Installation Guide}]{FPGA\_Vendor\_Tools\_Installation\_Guide.pdf}
% or if you want the raw ugly URL to come out, \githubioURL{FPGA_Vendor_Tools_Installation_Guide.pdf}
\newcommand{\githubio}[2][]{% The default is for FIRST param!
\href{http://opencpi.github.io/releases/\ocpiversion/#2}{\ifthenelse{\equal{#1}{}}{\texttt{#2}}{#1}}}
\newcommand{\githubioURL}[1]{\url{http://opencpi.github.io/releases/\ocpiversion/#1}}

% Fix import paths
\makeatletter
\def\input@path{{\snippetpath/}}
\makeatother

% From then on, you can use "input" With no paths to get to "snippets"
% You also get all "major" snippets not part of the global LaTeX_Header
% NOTE: If not using the global LaTeX_Header, you need to
% \usepackage{ifthen} to use the \githubio macro

\hyphenation{ANGRY-VIPER} % Tell it where to hyphenate
\hyphenation{Cent-OS} % Tell it where to hyphenate
\hyphenation{install-ation} % Tell it where to hyphenate

\newcommand{\todo}[1]{\textcolor{red}{TODO: #1}\PackageWarning{TODO:}{#1}} % To do notes
\newcommand{\code}[1]{\texttt{#1}} % For inline code snippet or command line
\newcommand{\sref}[1]{Section~\ref{#1}} % To quickly reference a section

% To quickly reference a versioned PDF on github.io
% \def\ocpiversion{1.5.0}
\def\ocpiversion{1.5.0rc4} % TEMPORARY

% This gives a link to github.io document. By default, it puts the filename.
% You can optionally change the link, e.g.
% \githubio{FPGA\_Vendor\_Tools\_Installation\_Guide.pdf} vs.
% \githubio[\textit{FPGA Vendor Tools Installation Guide}]{FPGA\_Vendor\_Tools\_Installation\_Guide.pdf}
% or if you want the raw ugly URL to come out, \githubioURL{FPGA_Vendor_Tools_Installation_Guide.pdf}
\newcommand{\githubio}[2][]{% The default is for FIRST param!
\href{http://opencpi.github.io/releases/\ocpiversion/#2}{\ifthenelse{\equal{#1}{}}{\texttt{#2}}{#1}}}
\newcommand{\githubioURL}[1]{\url{http://opencpi.github.io/releases/\ocpiversion/#1}}

% Fix import paths
\makeatletter
\def\input@path{{\snippetpath/}}
\makeatother

\documentclass{article}
\iffalse
This file is protected by Copyright. Please refer to the COPYRIGHT file
distributed with this source distribution.

This file is part of OpenCPI <http://www.opencpi.org>

OpenCPI is free software: you can redistribute it and/or modify it under the
terms of the GNU Lesser General Public License as published by the Free Software
Foundation, either version 3 of the License, or (at your option) any later
version.

OpenCPI is distributed in the hope that it will be useful, but WITHOUT ANY
WARRANTY; without even the implied warranty of MERCHANTABILITY or FITNESS FOR A
PARTICULAR PURPOSE. See the GNU Lesser General Public License for more details.

You should have received a copy of the GNU Lesser General Public License along
with this program. If not, see <http://www.gnu.org/licenses/>.
\fi
\author{} % Force author to be blank
%----------------------------------------------------------------------------------------
% Paper size, orientation and margins
%----------------------------------------------------------------------------------------
\usepackage{geometry}
\geometry{
        letterpaper, % paper type
        portrait,    % text direction
        left=.75in,  % left margin
        top=.75in,   % top margin
        right=.75in, % right margin
        bottom=.75in % bottom margin
 }
%----------------------------------------------------------------------------------------
% Header/Footer
%----------------------------------------------------------------------------------------
\usepackage{fancyhdr} \pagestyle{fancy} % required for fancy headers
\renewcommand{\headrulewidth}{0.5pt}
\renewcommand{\footrulewidth}{0.5pt}
\rhead{\small{ANGRYVIPER Team}}
% \rfoot{\thepage}
%----------------------------------------------------------------------------------------
% Appendix packages
%----------------------------------------------------------------------------------------
\usepackage[toc,page]{appendix}
%----------------------------------------------------------------------------------------
% Defined Commands & Renamed Commands
%----------------------------------------------------------------------------------------
\renewcommand{\contentsname}{Table of Contents}
\renewcommand{\listfigurename}{List of Figures}
\renewcommand{\listtablename}{List of Tables}
%----------------------------------------------------------------------------------------
% Various packages
%----------------------------------------------------------------------------------------
\usepackage[usenames,dvipsnames]{xcolor} % for color names see https://en.wikibooks.org/wiki/LaTeX/Colors
\usepackage{hyperref}  % for linking urls and lists
\usepackage{graphicx}  % for including pictures by file
\usepackage{listings}  % for coding language styles
\usepackage{rotating}  % for sideways table
\usepackage{pifont}    % for sideways table
\usepackage{pdflscape} % for landscape view
\usepackage{subfig}
\usepackage{xstring}
\uchyph=0 % Never hyphenate acronyms like RCC (I think this overrides ANGRYVIPER above)
\renewcommand\_{\textunderscore\allowbreak} % Allow words to break/newline on underscores
%----------------------------------------------------------------------------------------
% Table packages
%----------------------------------------------------------------------------------------
\usepackage{longtable} % for long possibly multi-page tables
\usepackage{tabularx} % c=center,l=left,r=right,X=fill
% These define tabularx columns "C" and "R" to match "X" but center/right aligned
\newcolumntype{C}{>{\centering\arraybackslash}X}
\newcolumntype{R}{>{\raggedleft\arraybackslash}X}
\usepackage{float}
\floatstyle{plaintop}
\usepackage[tableposition=top]{caption}
\newcolumntype{P}[1]{>{\centering\arraybackslash}p{#1}}
\newcolumntype{M}[1]{>{\centering\arraybackslash}m{#1}}
%----------------------------------------------------------------------------------------
% Block Diagram / FSM Drawings
%----------------------------------------------------------------------------------------
\usepackage{tikz}
\usetikzlibrary{shapes,arrows,fit,positioning}
\usetikzlibrary{automata} % used for the fsm
%----------------------------------------------------------------------------------------
% Colors Used
%----------------------------------------------------------------------------------------
\usepackage{colortbl}
\definecolor{blue}{rgb}{.7,.8,.9}
\definecolor{ceruleanblue}{rgb}{0.16, 0.32, 0.75}
\definecolor{drkgreen}{rgb}{0,0.6,0}
\definecolor{deepmagenta}{rgb}{0.8, 0.0, 0.8}
\definecolor{cyan}{rgb}{0.0,0.6,0.6}
\definecolor{maroon}{rgb}{0.5,0,0}
%----------------------------------------------------------------------------------------
% VHDL Coding Language Style
% modified from: http://latex-community.org/forum/viewtopic.php?f=44&t=22076
%----------------------------------------------------------------------------------------
\lstdefinelanguage{VHDL}
{
        basicstyle=\ttfamily\footnotesize,
        columns=fullflexible,keepspaces,      % https://tex.stackexchange.com/a/46695/87531
        keywordstyle=\color{ceruleanblue},
        commentstyle=\color{drkgreen},
        morekeywords={
    library,use,all,entity,is,port,in,out,end,architecture,of,
    begin,and, signal, when, if, else, process, end,
        },
        morecomment=[l]--
}
%----------------------------------------------------------------------------------------
% XML Coding Language Style
% modified from: http://tex.stackexchange.com/questions/10255/xml-syntax-highlighting
%----------------------------------------------------------------------------------------
\lstdefinelanguage{XML}
{
        basicstyle=\ttfamily\footnotesize,
        columns=fullflexible,keepspaces,
        morestring=[s]{"}{"},
        morecomment=[s]{!--}{--},
        commentstyle=\color{drkgreen},
        moredelim=[s][\color{black}]{>}{<},
        moredelim=[s][\color{cyan}]{\ }{=},
        stringstyle=\color{maroon},
        identifierstyle=\color{ceruleanblue}
}
%----------------------------------------------------------------------------------------
% DIFF Coding Language Style
% modified from http://tex.stackexchange.com/questions/50176/highlighting-a-diff-file
%----------------------------------------------------------------------------------------
\lstdefinelanguage{diff}
{
        basicstyle=\ttfamily\footnotesize,
        columns=fullflexible,keepspaces,
        breaklines=true,                                % wrap text
        morecomment=[f][\color{ceruleanblue}]{@@},      % group identifier
        morecomment=[f][\color{red}]-,                  % deleted lines
        morecomment=[f][\color{drkgreen}]+,             % added lines
        morecomment=[f][\color{deepmagenta}]{---},      % Diff header lines (must appear after +,-)
        morecomment=[f][\color{deepmagenta}]{+++},
}
%----------------------------------------------------------------------------------------
% Python Coding Language Style
% modified from
%----------------------------------------------------------------------------------------
\lstdefinelanguage{python}
{
        basicstyle=\ttfamily\footnotesize,
        columns=fullflexible,keepspaces,
        keywordstyle=\color{ceruleanblue},
        commentstyle=\color{drkgreen},
        stringstyle=\color{orange},
        morekeywords={
    print, if, sys, len, from, import, as, open,close, def, main, for, else, write, read, range,
        },
        comment=[l]{\#}
}
%----------------------------------------------------------------------------------------
% Fontsize Notes in order from smallest to largest
%----------------------------------------------------------------------------------------
%    \tiny
%    \scriptsize
%    \footnotesize
%    \small
%    \normalsize
%    \large
%    \Large
%    \LARGE
%    \huge
%    \Huge

\date{Version \docVersion} % Force date to be blank and override date with version
\title{\docTitle}
\lhead{FSK App RCC Getting Started Guide}
\setcounter{secnumdepth}{4}
\setcounter{tocdepth}{4}
\newcommand{\subsubsubsection}[1]{\paragraph{#1}\mbox{}\\\mbox{}\\}
%----------------------------------------------------------------------------------------
\usepackage[T1]{fontenc} % http://tex.stackexchange.com/a/181119
\usepackage{graphicx}
\graphicspath{ {figures/} }
\lstset{ % https://tex.stackexchange.com/a/116572
  basicstyle=\ttfamily,
  columns=fullflexible,
  % frame=single,
  breaklines=true,
  showstringspaces=true,
  showspaces=true,
  postbreak=\mbox{\textcolor{red}{$\hookrightarrow$}\space},
}
\usepackage{textcomp}
\usepackage{enumerate}
\begin{document}
\maketitle
%\thispagestyle{fancy}
\newpage

	\begin{center}
	\textit{\textbf{Revision History}}
		\begin{table}[H]
		\label{table:revisions} % Add "[H]" to force placement of table
			\begin{tabularx}{\textwidth}{|c|X|l|}
			\hline
			\rowcolor{blue}
			\textbf{Revision} & \textbf{Description of Change} & \textbf{Date} \\
		    \hline
		    v1.6 & Initial Release, FSK App RCC only mode & 8/2019 \\
			\hline
			\end{tabularx}
		\end{table}
	\end{center}

\newpage

\tableofcontents

\newpage

\section{References}

	This document assumes a basic understanding of the Linux command line (or ``shell'') environment.  The reference(s) in Table 1 can be used as an overview of OpenCPI and may prove useful.

\def\myreferences{
\hline
FSK App Guide\footnote{Provides details of the ``FSK App'' reference application used in this document} & \githubioFlat{assets/FSK\_app.pdf} \\
\hline
FSK App Getting Started Guide & \githubioFlat{assets/FSK\_App\_RCC\_Getting\_Started\_Guide.pdf} \\
}
\iffalse
This file is protected by Copyright. Please refer to the COPYRIGHT file
distributed with this source distribution.

This file is part of OpenCPI <http://www.opencpi.org>

OpenCPI is free software: you can redistribute it and/or modify it under the
terms of the GNU Lesser General Public License as published by the Free Software
Foundation, either version 3 of the License, or (at your option) any later
version.

OpenCPI is distributed in the hope that it will be useful, but WITHOUT ANY
WARRANTY; without even the implied warranty of MERCHANTABILITY or FITNESS FOR A
PARTICULAR PURPOSE. See the GNU Lesser General Public License for more details.

You should have received a copy of the GNU Lesser General Public License along
with this program. If not, see <http://www.gnu.org/licenses/>.
\fi

% This snippet creates the "References" table labeled "table:references"
% It creates three columns: Name, Publisher, Link and then inserts default documents
%
% To skip these defaults, define macros named
% refskipgs to skip "Getting Started"
% refskipig to skip "Installation Guide"
% refskipac to skip "Acronyms and Definitions"
% refskipocpiov to skip "OpenCPI Overview"
%
% See RPM_Installation_Guide.tex for examples
%
% After the defaults, it optionally inserts the "myreferences" macro that
% you defined elsewhere (you put hlines above all lines)
%
% If you want the \caption on the bottom, define "refcapbottom"
\begin{center}
\renewcommand*\footnoterule{} % Remove separator line from footnote
\renewcommand{\thempfootnote}{\arabic{mpfootnote}} % Use Arabic numbers (or can't reuse)
\begin{minipage}{0.9\textwidth}
  \begin{table}[H]
\ifx\refcapbottom\undefined
  \caption {References}
  \label{table:references}
\fi
  \begin{tabularx}{\textwidth}{|C|C|}
    \hline
    \rowcolor{blue}
    \textbf{Title} & \textbf{Link} \\
\ifx\refskipocpiov\undefined
    \hline
    OpenCPI Overview & \githubio{Overview.pdf} \\
\fi
\ifx\refskipac\undefined
    \hline
    Acronyms and Definitions & \githubio{Acronyms\_and\_Definitions.pdf} \\
\fi
\ifx\refskipgs\undefined
    \hline
    Getting Started & \githubio{Getting\_Started.pdf} \\
\fi
\ifx\refskipig\undefined
    \hline
    Installation Guide & \githubio{RPM\_Installation\_Guide.pdf} \\
\fi
\ifx\myreferences\undefined
\else
    \myreferences
\fi
    \hline
  \end{tabularx}
\ifx\refcapbottom\undefined
\else
  \caption {References}
  \label{table:references}
\fi
  \end{table}
\end{minipage}
\end{center}


\newpage
\begin{flushleft}
\section{Overview}
This purpose of this document is to provide a compact set of instructions to build, run, and verify the OpenCPI Frequency Shift Keying Modulation (FSK) reference application (known as the ``FSK App'') in a software-only manner.
While the FSK reference application supports several heterogeneous execution modes (\textit{filerw}, \textit{txrx}, \textit{bbloopback}, \textit{rx}, \textit{tx}) and is supported on various systems, this guide will only discuss execution of the \textit{filerw\_rcc} mode on CentOS~7 and/or Epiq Solution's Matchstiq-Z1.
These steps can be adjusted for other platforms. As its name implies, the \textit{filerw\_rcc} mode of the FSK App operates \textit{only} using RCC workers.
~\\ \medskip
Successful execution of the FSK App in the \textit{filerw\_rcc} mode demonstrates that the framework has been built successfully and also validates that an RCC platform's workers and framework libraries properly execute.
Additionally, when developing the RCC portion of a Board Support Package (BSP), this allows independently validating that the RCC platform works as a first step towards a fully-featured BSP.
The HDL platform can be later validated using one of the other FSK App heterogeneous modes in the companion document \githubio[\textit{FSK App Getting Started Guide}]{assets/FSK\_App\_Getting\_Started\_Guide.pdf}.
~\\ \medskip
The definitive documentation for the FSK App, including the remaining modes, is the \githubio[\textit{FSK App Guide}]{assets/FSK\_app.pdf}.

\section{Prerequisites}
This document assumes that the OpenCPI framework has been installed.  The application is supported on all of the OpenCPI RCC platforms, but this document shows examples for CentOS~7's platform \path{centos7} and Matchstiq-Z1's platform \path{xilinx13_3}.

\section{Building the FSK Application }
This section explains the steps for building everything to run the FSK App.  Following these steps will build more workers than required for the FSK App to execute, but provides an easy way to be certain all artifacts have been built.
% Since this is repeated a lot... the ugliness with {-}- stops TeX from making it a longer dash
\newcommand{\BuildRCCForPlatform}[1]{\texttt{ocpidev build {-}-rcc {-}-rcc-platform #1}}
\newcommand{\BuildProject}[1]{
\subsection{Build the \textit{#1} project}
If the \textit{#1} project has not been created, follow the instructions in the Getting Started Guide.
Once the \textit{#1} project has been created, the following \path{ocpidev} command can be used to build the primitives and workers required by the FSK App.
Navigate to the installed and registered \textit{#1} project directory and run the command:
\begin{itemize}
\item CentOS~7:
\subitem\BuildRCCForPlatform{centos7}
\item Matchstiq-Z1:
\subitem\BuildRCCForPlatform{xilinx13\_3}
\end{itemize}
}
% Output the two sections:
\foreach \proj in {core, assets} {
  \BuildProject{\proj}
}
% App Building doesn't use the "--rcc" flag (WHY?)
\newcommand{\BuildAppForPlatform}[1]{\texttt{ocpidev build {-}-rcc-platform #1}}
\subsection{Build the FSK Application Executable}
Next, ensure the executable for the FSK Application has been built. Navigate to the \path{applications/FSK} of the \textit{assets} project and run the command:
\begin{itemize}
\item CentOS~7:
\subitem\BuildAppForPlatform{centos7}
\subitem If successful, an executable file \path{target-centos7/FSK} should now exist.
\item Matchstiq-Z1:
\subitem\BuildAppForPlatform{xilinx13\_3}
\subitem If successful, an executable file \path{target-xilinx13_3/FSK} should now exist.
\end{itemize}

\section{Running the Application on CentOS~7}
This example demonstrates running the framework and FSK App on a host machine that was used to build the artifacts.
The following steps describe setup and execution of the FSK in \textit{filerw\_rcc} mode on
CentOS~7.

\subsection{Setting OCPI\_LIBRARY\_PATH Environment Variable}
\newcommand{\OcpiLibraryPathBlurb}{
The \path{OCPI_LIBRARY_PATH} environment variable is a colon-separated list of files/directories which is searched for executable artifacts during deployment of the application.
For proper execution of the FSK App, all artifacts (detailed in the FSK App document) must be included in the \path{OCPI_LIBRARY_PATH} environment variable.
Example syntax for setting the library path is provided below.}
\OcpiLibraryPathBlurb~\\
\begin{verbatim}
$ export OCPI_LIBRARY_PATH=~/ocpi_projects/assets/artifacts/:~/ocpi_projects/core/artifacts/
\end{verbatim}
\subsection{Running the Application}
To run the application, navigate to the \path{assets/applications/FSK} directory and run the \path{FSK} executable in the \path{target-centos7} directory. Example syntax can be seen below.
\begin{verbatim}
$ ./target-centos7/FSK filerw_rcc
\end{verbatim}

After execution, continue to \sref{ViewTheResults}.

\section{Running FSK App on Matchstiq-Z1 }
The following steps describe setup and execution of the FSK in \textit{filerw\_rcc} mode on the Matchstiq-Z1.  This section requires that the Matchstiq-Z1 has been prepared for use with OpenCPI, ensure that the \githubio[\textit{Matchstiq\_Z1\ Getting Started Guide}]{assets/Matchstiq\_Z1\_Getting\_Started\_Guide.pdf} has been completed and network mode is functional before proceeding.
Running the FSK App on any non-host platform demonstrates that the framework and RCC platform configuration (cross-compilation flags, etc.) work correctly.
Additionally, these steps can be used to run the application with the framework artifacts stored remotely on a network store (networked mode) or located on the device (standalone mode).

\subsection{Setting Up the Execution Platform}
For embedded platforms (\textit{e.g.} Matchstiq-Z1), connect to the platform via a serial port. Example syntax for establishing a connection can be seen below:

\begin{itemize}
\item Matchstiq-Z1\footnote{This command works only if the \path{udev} rules have been properly loaded.}:
\begin{itemize}
\item[]
\begin{verbatim}
screen /dev/matchstiq_z1_0 115200
\end{verbatim}
\end{itemize}
\end{itemize}

\begin{itemize}
\item In General:
\begin{itemize}
\item[]
\begin{verbatim}
sudo screen /dev/ttyUSB0 115200
\end{verbatim}
\end{itemize}
\end{itemize}

For this document we use what is called Networked mode, which is one of two modes typically used for running applications\footnote{Consult the \textit{Matchstiq-Z1 Getting Started Guide} for more information}. It involves creating NFS mounts between the development platform and the execution platform to enable application deployment. A setup script is used to automate the required steps for this mode. The example script must be modified to match your host machine's paths before you run it. Example syntax for running this script can be seen below:
\begin{verbatim}
source /mnt/card/opencpi/mynetsetup.sh <IP Address of Development Host>
\end{verbatim}
\subsection{Setting OCPI\_LIBRARY\_PATH Environment Variable}
\OcpiLibraryPathBlurb~\\
\begin{verbatim}
$ export OCPI_LIBRARY_PATH=/mnt/ocpi_assets/artifacts:/mnt/ocpi_core/artifacts
\end{verbatim}

\subsection{Running the Application}
To run the application, navigate to the \path{applications/FSK} directory and run the executable in the \path{target-<RCC Platform>} directory of your Execution Platform. Example syntax can be seen below.
\begin{verbatim}
$ ./target-xilinx13_3/FSK filerw_rcc
\end{verbatim}
Reference the \githubio[\textit{FSK App Getting Started Guide}]{assets/FSK\_App\_Getting\_Started\_Guide.pdf} for executing the remaining modes (\textit{filerw}, \textit{txrx}, \textit{bbloopback}, \textit{rx}, \textit{tx}) of the FSK Application. Ensure that \path{OCPI_LIBRARY_PATH} is configured, as provided, for the respective mode and platform. \\ \medskip

\section{View the Results}\label{ViewTheResults}
After the application completes, the output can be found in the \path{applications/FSK/odata} directory. To view the results on the Development Host, navigate to the \path{applications/FSK} directory and verify the output of these commands:
% AV-5637: New image
\begin{verbatim}
$ file odata/out_app_fsk_filerw_rcc.bin
odata/out_app_fsk_filerw_rcc.bin: JPEG image data, JFIF standard 1.01
$ eog odata/out_app_fsk_filerw_rcc.bin
<Image Viewer launches, showing image of Orioles Bird>
\end{verbatim}
\end{flushleft}
\end{document}
