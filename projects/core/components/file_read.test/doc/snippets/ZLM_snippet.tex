\subsubsection*{Zero Length Messages}
A zero-length message (ZLM) is any message that has zero payload (data) associated with it but the rest of the message signaling and opcode are passed through.  The default operation for end-of-file for File\_Read is to pass along a ZLM on the opcode set by the property \textit{opcode} (default is 0) which is then advanced to the downstream worker. The ZLM is processed by the File\_Write component at the end of the application chain and File\_Write sets its state to``done''.  In file based applications it is common practice to set an application to complete when the File write component is done to ensure all data is processed.  This means that all other workers in the path must process and pass on ZLMs for this to function properly.
