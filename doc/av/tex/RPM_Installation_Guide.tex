\iffalse
This file is protected by Copyright. Please refer to the COPYRIGHT file
distributed with this source distribution.

This file is part of OpenCPI <http://www.opencpi.org>

OpenCPI is free software: you can redistribute it and/or modify it under the
terms of the GNU Lesser General Public License as published by the Free Software
Foundation, either version 3 of the License, or (at your option) any later
version.

OpenCPI is distributed in the hope that it will be useful, but WITHOUT ANY
WARRANTY; without even the implied warranty of MERCHANTABILITY or FITNESS FOR A
PARTICULAR PURPOSE. See the GNU Lesser General Public License for more details.

You should have received a copy of the GNU Lesser General Public License along
with this program. If not, see <http://www.gnu.org/licenses/>.
\fi
 %----------------------------------------------------------------------------------------
% Update the docTitle and docVersion per document
%----------------------------------------------------------------------------------------
\def\docTitle{RPM Installation Guide}
\def\docVersion{1.4}
%----------------------------------------------------------------------------------------
\documentclass{article}
\iffalse
This file is protected by Copyright. Please refer to the COPYRIGHT file
distributed with this source distribution.

This file is part of OpenCPI <http://www.opencpi.org>

OpenCPI is free software: you can redistribute it and/or modify it under the
terms of the GNU Lesser General Public License as published by the Free Software
Foundation, either version 3 of the License, or (at your option) any later
version.

OpenCPI is distributed in the hope that it will be useful, but WITHOUT ANY
WARRANTY; without even the implied warranty of MERCHANTABILITY or FITNESS FOR A
PARTICULAR PURPOSE. See the GNU Lesser General Public License for more details.

You should have received a copy of the GNU Lesser General Public License along
with this program. If not, see <http://www.gnu.org/licenses/>.
\fi
\author{} % Force author to be blank
%----------------------------------------------------------------------------------------
% Paper size, orientation and margins
%----------------------------------------------------------------------------------------
\usepackage{geometry}
\geometry{
        letterpaper, % paper type
        portrait,    % text direction
        left=.75in,  % left margin
        top=.75in,   % top margin
        right=.75in, % right margin
        bottom=.75in % bottom margin
 }
%----------------------------------------------------------------------------------------
% Header/Footer
%----------------------------------------------------------------------------------------
\usepackage{fancyhdr} \pagestyle{fancy} % required for fancy headers
\renewcommand{\headrulewidth}{0.5pt}
\renewcommand{\footrulewidth}{0.5pt}
\rhead{\small{ANGRYVIPER Team}}
% \rfoot{\thepage}
%----------------------------------------------------------------------------------------
% Appendix packages
%----------------------------------------------------------------------------------------
\usepackage[toc,page]{appendix}
%----------------------------------------------------------------------------------------
% Defined Commands & Renamed Commands
%----------------------------------------------------------------------------------------
\renewcommand{\contentsname}{Table of Contents}
\renewcommand{\listfigurename}{List of Figures}
\renewcommand{\listtablename}{List of Tables}
%----------------------------------------------------------------------------------------
% Various packages
%----------------------------------------------------------------------------------------
\usepackage[usenames,dvipsnames]{xcolor} % for color names see https://en.wikibooks.org/wiki/LaTeX/Colors
\usepackage{hyperref}  % for linking urls and lists
\usepackage{graphicx}  % for including pictures by file
\usepackage{listings}  % for coding language styles
\usepackage{rotating}  % for sideways table
\usepackage{pifont}    % for sideways table
\usepackage{pdflscape} % for landscape view
\usepackage{subfig}
\usepackage{xstring}
\uchyph=0 % Never hyphenate acronyms like RCC (I think this overrides ANGRYVIPER above)
\renewcommand\_{\textunderscore\allowbreak} % Allow words to break/newline on underscores
%----------------------------------------------------------------------------------------
% Table packages
%----------------------------------------------------------------------------------------
\usepackage{longtable} % for long possibly multi-page tables
\usepackage{tabularx} % c=center,l=left,r=right,X=fill
% These define tabularx columns "C" and "R" to match "X" but center/right aligned
\newcolumntype{C}{>{\centering\arraybackslash}X}
\newcolumntype{R}{>{\raggedleft\arraybackslash}X}
\usepackage{float}
\floatstyle{plaintop}
\usepackage[tableposition=top]{caption}
\newcolumntype{P}[1]{>{\centering\arraybackslash}p{#1}}
\newcolumntype{M}[1]{>{\centering\arraybackslash}m{#1}}
%----------------------------------------------------------------------------------------
% Block Diagram / FSM Drawings
%----------------------------------------------------------------------------------------
\usepackage{tikz}
\usetikzlibrary{shapes,arrows,fit,positioning}
\usetikzlibrary{automata} % used for the fsm
%----------------------------------------------------------------------------------------
% Colors Used
%----------------------------------------------------------------------------------------
\usepackage{colortbl}
\definecolor{blue}{rgb}{.7,.8,.9}
\definecolor{ceruleanblue}{rgb}{0.16, 0.32, 0.75}
\definecolor{drkgreen}{rgb}{0,0.6,0}
\definecolor{deepmagenta}{rgb}{0.8, 0.0, 0.8}
\definecolor{cyan}{rgb}{0.0,0.6,0.6}
\definecolor{maroon}{rgb}{0.5,0,0}
%----------------------------------------------------------------------------------------
% VHDL Coding Language Style
% modified from: http://latex-community.org/forum/viewtopic.php?f=44&t=22076
%----------------------------------------------------------------------------------------
\lstdefinelanguage{VHDL}
{
        basicstyle=\ttfamily\footnotesize,
        columns=fullflexible,keepspaces,      % https://tex.stackexchange.com/a/46695/87531
        keywordstyle=\color{ceruleanblue},
        commentstyle=\color{drkgreen},
        morekeywords={
    library,use,all,entity,is,port,in,out,end,architecture,of,
    begin,and, signal, when, if, else, process, end,
        },
        morecomment=[l]--
}
%----------------------------------------------------------------------------------------
% XML Coding Language Style
% modified from: http://tex.stackexchange.com/questions/10255/xml-syntax-highlighting
%----------------------------------------------------------------------------------------
\lstdefinelanguage{XML}
{
        basicstyle=\ttfamily\footnotesize,
        columns=fullflexible,keepspaces,
        morestring=[s]{"}{"},
        morecomment=[s]{!--}{--},
        commentstyle=\color{drkgreen},
        moredelim=[s][\color{black}]{>}{<},
        moredelim=[s][\color{cyan}]{\ }{=},
        stringstyle=\color{maroon},
        identifierstyle=\color{ceruleanblue}
}
%----------------------------------------------------------------------------------------
% DIFF Coding Language Style
% modified from http://tex.stackexchange.com/questions/50176/highlighting-a-diff-file
%----------------------------------------------------------------------------------------
\lstdefinelanguage{diff}
{
        basicstyle=\ttfamily\footnotesize,
        columns=fullflexible,keepspaces,
        breaklines=true,                                % wrap text
        morecomment=[f][\color{ceruleanblue}]{@@},      % group identifier
        morecomment=[f][\color{red}]-,                  % deleted lines
        morecomment=[f][\color{drkgreen}]+,             % added lines
        morecomment=[f][\color{deepmagenta}]{---},      % Diff header lines (must appear after +,-)
        morecomment=[f][\color{deepmagenta}]{+++},
}
%----------------------------------------------------------------------------------------
% Python Coding Language Style
% modified from
%----------------------------------------------------------------------------------------
\lstdefinelanguage{python}
{
        basicstyle=\ttfamily\footnotesize,
        columns=fullflexible,keepspaces,
        keywordstyle=\color{ceruleanblue},
        commentstyle=\color{drkgreen},
        stringstyle=\color{orange},
        morekeywords={
    print, if, sys, len, from, import, as, open,close, def, main, for, else, write, read, range,
        },
        comment=[l]{\#}
}
%----------------------------------------------------------------------------------------
% Fontsize Notes in order from smallest to largest
%----------------------------------------------------------------------------------------
%    \tiny
%    \scriptsize
%    \footnotesize
%    \small
%    \normalsize
%    \large
%    \Large
%    \LARGE
%    \huge
%    \Huge

\date{Version \docVersion} % Force date to be blank and override date with version
\title{\docTitle}
\lhead{\small{\docTitle}}
\setlength{\parindent}{0pt} % Don't indent all paragraphs
\newcommand{\forceindent}{\leavevmode{\parindent=1em\indent}}
% This block is to make sure there is 3cm min at the bottom of a page before a new section or subsection is allowed to start. Otherwise, next page.
% Modified from http://tex.stackexchange.com/a/152278
\usepackage{etoolbox}
\newskip\mfilskip
\mfilskip=0pt plus 3cm\relax
\newcommand{\mfilbreak}{\vspace{\mfilskip}\penalty -200%
  \ifdim\lastskip<\mfilskip\vspace{-\lastskip}\else\vspace{-\mfilskip}\fi}
\pretocmd{\section}{\mfilbreak}{}{}
\pretocmd{\subsection}{\mfilbreak}{}{}
% end [sub]section pushes
%----------------------------------------------------------------------------------------
\begin{document}
\maketitle
\thispagestyle{fancy}
\newpage

	\begin{center}
	\textit{\textbf{Revision History}}
		\begin{table}[H] % Add "[H]" to force placement of table
		\label{table:revisions}
		\centering
			\begin{tabularx}{.7\textwidth}{|c|X|l|}
			\hline
			\rowcolor{blue}
			\textbf{Revision} & \textbf{Description of Change} & \textbf{Date} \\
		    \hline
			v1.0 & Initial Release & 2/2016 \\
		    \hline
			v1.1 & Updated for OpenCPI Release 1.1 & 3/2017 \\
			\hline
			v1.2 & Updated for OpenCPI Release 1.2 & 8/2017 \\
			\hline
			v1.3 & Updated for OpenCPI Release 1.3 & 2/2018 \\
			\hline
			v1.3.1 & Updated for OpenCPI Release 1.3.1 & 4/2018 \\
			\hline
			v1.4 & Updated for OpenCPI Release 1.4 & 9/2018 \\
			\hline
			\end{tabularx}
		\end{table}
	\end{center}

\newpage

\tableofcontents

% \newpage

% \listoffigures
% \newpage

\listoftables

\newpage

\section{References}
This document assumes a basic understanding of the Linux command line environment. It does not require a working knowledge of OpenCPI. However, it is recommended that the user read the \textit{Getting Started} document (up to the ``Installation of OpenCPI'' section) or reference the \textit{Acronyms and Definitions} document for various terms used within.
\def\refskipig{} % Skip self
\def\myreferences{
\hline
Installation Guide\footnote{The RPM installation process is quite different from the process explained in the OpenCPI Installation Guide, but the OpenCPI Installation guide has applicable post-installation information for PCI-based boards, etc.}
& OpenCPI & \url{https://opencpi.github.io/OpenCPI_Installation.pdf} \\
\hline
Component Development Guide & OpenCPI & \url{https://opencpi.github.io/OpenCPI_Component_Development.pdf} \\
\hline
RCC Development Guide & OpenCPI & \url{https://opencpi.github.io/OpenCPI_RCC_Development.pdf} \\
\hline
HDL Development Guide & OpenCPI & \url{https://opencpi.github.io/OpenCPI_HDL_Development.pdf} \\
\hline
FPGA Vendor Tools Installation Guide & ANGRYVIPER & \url{https://opencpi.github.io/FPGA_Vendor_Tools_Installation_Guide.pdf} \\
\hline
Managing Software with \texttt{yum} & CentOS Project &  \url{https://www.centos.org/docs/5/html/yum/} \\
\hline
CentOS Deployment Guide: Useful \texttt{yum} commands (\textit{e.g.} \texttt{yum localinstall}) & CentOS Project &  \url{https://www.centos.org/docs/5/html/5.2/Deployment_Guide/s1-yum-useful-commands.html} \\
}
\iffalse
This file is protected by Copyright. Please refer to the COPYRIGHT file
distributed with this source distribution.

This file is part of OpenCPI <http://www.opencpi.org>

OpenCPI is free software: you can redistribute it and/or modify it under the
terms of the GNU Lesser General Public License as published by the Free Software
Foundation, either version 3 of the License, or (at your option) any later
version.

OpenCPI is distributed in the hope that it will be useful, but WITHOUT ANY
WARRANTY; without even the implied warranty of MERCHANTABILITY or FITNESS FOR A
PARTICULAR PURPOSE. See the GNU Lesser General Public License for more details.

You should have received a copy of the GNU Lesser General Public License along
with this program. If not, see <http://www.gnu.org/licenses/>.
\fi

% This snippet creates the "References" table labeled "table:references"
% It creates three columns: Name, Publisher, Link and then inserts default documents
%
% To skip these defaults, define macros named
% refskipgs to skip "Getting Started"
% refskipig to skip "Installation Guide"
% refskipac to skip "Acronyms and Definitions"
% refskipocpiov to skip "OpenCPI Overview"
%
% See RPM_Installation_Guide.tex for examples
%
% After the defaults, it optionally inserts the "myreferences" macro that
% you defined elsewhere (you put hlines above all lines)
%
% If you want the \caption on the bottom, define "refcapbottom"
\begin{center}
\renewcommand*\footnoterule{} % Remove separator line from footnote
\renewcommand{\thempfootnote}{\arabic{mpfootnote}} % Use Arabic numbers (or can't reuse)
\begin{minipage}{0.9\textwidth}
  \begin{table}[H]
\ifx\refcapbottom\undefined
  \caption {References}
  \label{table:references}
\fi
  \begin{tabularx}{\textwidth}{|C|C|}
    \hline
    \rowcolor{blue}
    \textbf{Title} & \textbf{Link} \\
\ifx\refskipocpiov\undefined
    \hline
    OpenCPI Overview & \githubio{Overview.pdf} \\
\fi
\ifx\refskipac\undefined
    \hline
    Acronyms and Definitions & \githubio{Acronyms\_and\_Definitions.pdf} \\
\fi
\ifx\refskipgs\undefined
    \hline
    Getting Started & \githubio{Getting\_Started.pdf} \\
\fi
\ifx\refskipig\undefined
    \hline
    Installation Guide & \githubio{RPM\_Installation\_Guide.pdf} \\
\fi
\ifx\myreferences\undefined
\else
    \myreferences
\fi
    \hline
  \end{tabularx}
\ifx\refcapbottom\undefined
\else
  \caption {References}
  \label{table:references}
\fi
  \end{table}
\end{minipage}
\end{center}

\newpage
\section{Document Overview}
\label{sec:doc_overview}
This document describes how to \textbf{install OpenCPI at a system level} on a development host for multiple users via RPMs. The host installation allows for local software-based execution of OpenCPI applications and components, cross-building for non-x86 platforms, simulation of HDL, and, when available, hardware testing. \textbf{Upon completion of this Guide, the steps described in the \textit{Getting Started Guide} must be followed by \textit{each} OpenCPI user.}\\

The default host installation platform for OpenCPI development is CentOS~6 or CentOS~7 Linux x86\_64 (64-bit). Other Linux variants and 32-bit systems have been used successfully, but this document expects the OS to be CentOS~7. Development hosts can either be actual physical systems or virtual machine installations.\\
\begin{center}
\framebox{\parbox{0.8\linewidth}{\centering This document assumes that CentOS is already installed and\\ proper administrative privileges have been established.}}
\end{center}
\ \\ % Gap under box
Additional installation options exist for other target processors and technologies such as the Xilinx Zynq SoC (with ARM processor cores and FPGA resources). Preference when targeting non-x86 architectures is given to \textit{cross-building}, rather than self-hosting development. This limits the complexity of installing tools on different development hosts.\\

Installation of OpenCPI is completed in the following steps:
\begin{enumerate}
\item\textbf{Section~\ref{sec:acquiring_opencpi}}: Acquiring the OpenCPI framework
\item\textbf{Section~\ref{sec:install_opencpi}}: Installing the OpenCPI framework
\item\textbf{Section~\ref{sec:setup_opencpi}}: Setting up the OpenCPI environment
\item\textbf{Section~\ref{sec:testing_opencpi}}: Testing the OpenCPI installation
\end{enumerate}
These steps result in a development system with tooling and runtime software ready to support development and native execution of OpenCPI components and applications.

\section{Acquiring the OpenCPI framework}
\label{sec:acquiring_opencpi}
Currently, the ANGRYVIPER Team releases DVD-Rs containing the RPMs and PDF documentation of the OpenCPI framework. These are also available on \url{https://opencpi.github.io/}.

\section{Installing OpenCPI framework}
\label{sec:install_opencpi}
The ANGRYVIPER Team's recommended installation method for development is through the use of RPMs. The framework can be built from source for a development host, but is not recommended.\\

The ANGRYVIPER Team provides RPMs to their direct customers and other users can find them on \path{github.io}.
% While discouraged, documentation concerning building the RPMs from scratch can be found in \path{packaging/README}.

\subsubsection*{Understanding OpenCPI RPM naming convention}
\label{sec:understand_rpm_naming}
OpenCPI's RPM naming follows that of the Red Hat Package Manager recommendations of \path{<name>-<version>-<release>.<dist>.<architecture>.rpm} where:

\begin{enumerate}
	\setcounter{enumi}{0} % previous number list number
 	\item \textit{name} is the name describing the packaged software
 	\item \textit{version} is the version of the packaged software
 	\begin{enumerate}
	 	\item version following the Major.Minor.Sub-minor naming schema
	\end{enumerate}
 	\item \textit{release} is the number of times this version of software has been packaged
	\begin{enumerate}
	 	\item{this number is independent of the version}
	 \end{enumerate}
 	\item \textit{dist} is the OS distribution that the package is built for (\textit{e.g.} \texttt{.el7.centos})
 	\item \textit{architecture} is shorthand name describing the type of hardware the packaged software is to be installed on
 	\item ``\textit{devel}'' is sometimes appended to the package's name to indicate development RPMs which are required for building from source
\end{enumerate}

\subsubsection*{When to Install}
It is recommended that the user install these packages \textit{before} additional tools described in Section~\ref{subsec:installing_fpga_vendor} because the RPMs force the installation of some otherwise-hidden dependencies, \textit{e.g.} 32-bit X11 libraries for ModelSim.

\subsection{Installing OpenCPI from RPMs}
\label{sec:install_av_rpm}
It is recommended that the user installs all available packages whenever possible. If limited by available disk space, Table~\ref{table:decide} can be used to help determine which of the packages should be installed based upon the intended use of the target machine.\\

Within OpenCPI, there are two types of implementations, called \textit{Workers}, that are used in this framework: Resource-Constrained C Language (RCC) Workers and Hardware Description Language (HDL) Workers. RCC Workers are written using either C or C++ and are designed for either x86 or ARM architecture, while HDL Workers are written in VHDL and are designed for Field Programmable Gate Arrays (FPGAs) or HDL Simulators. For further details regarding RCC and HDL Workers see the \textit{OpenCPI RCC Development Guide} and the \textit{OpenCPI HDL Development Guide} (cf. Table~\ref{table:references}).
\begin{center}
\begin{minipage}{.75\textwidth}
	% Make new row command - single space is unchecked; anything else checks
	\newcommand{\rpm}[6]{\multicolumn{1}{|r|}{\texttt{#1}} &
		\ifthenelse{ \equal{#2}{ } }{}{\ding{51}} &
		\ifthenelse{ \equal{#3}{ } }{}{\ding{51}} &
		\ifthenelse{ \equal{#4}{ } }{}{\ding{51}} &
		\ifthenelse{ \equal{#5}{ } }{}{\ding{51}} &
		\ifthenelse{ \equal{#6}{ } }{}{\ding{51}}\\\hline}
	\renewcommand*\footnoterule{} % Remove separator line from footnote
	\renewcommand{\thempfootnote}{\arabic{mpfootnote}} % Use Arabic numbers (or can't reuse)
	\begin{table}[H]
	\caption{RPM Decision Guide}
	\label{table:decide}
	\begin{tabular}{r|c|c|c|c|c|}
		\cline{2-6}
		&\begin{turn}{90}Runtime RCC Host\end{turn}
		&\begin{turn}{90}Runtime HDL Host\end{turn}
		&\begin{turn}{90}RCC-Only Development\end{turn}\newline\begin{turn}{90}(x86 RCC exclusive)\end{turn}
		&\begin{turn}{90}RCC/HDL Development\end{turn}\newline\begin{turn}{90}(x86 RCC, non-SoC\footnote{``Non-SoC'' meaning a standalone FPGA \textit{without} an integrated processor, \textit{e.g.} Xilinx ML605.} FPGA HDL)\end{turn}
		&\begin{turn}{90}RCC/HDL Development\end{turn}\newline\begin{turn}{90}(Targeting non-x86 HW/SW platform)\end{turn}\\\hline
		\rpm{angryviper-ide...rpm   }{ }{ }{.}{.}{.}
		\rpm{opencpi-...rpm         }{x}{x}{x}{x}{x}
		\rpm{opencpi-debuginfo...rpm}{ }{ }{x}{x}{x}
		\rpm{opencpi-devel...rpm    }{ }{ }{x}{x}{x}
		\rpm{opencpi-doc...rpm      }{ }{ }{x}{x}{x}
		\rpm{opencpi-driver...rpm   }{ }{x}{ }{x}{x}
		\rpm{opencpi-project-bsp...rpm\footnote{BSP RPMs may not be provided with the standard/basic RPMs, but represent a placeholder for RPMs providing Board Support Package Projects.% There are certain BSPs which are located in the \code{ocpi.assets} Project and therefore do not require their own separate BSP RPMs.
		}}{ }{ }{x}{x}{x}
		\rpm{opencpi-*-platform...rpm}{ }{ }{ }{ }{x}
	\end{tabular}
	\end{table}
\end{minipage}
\end{center}

The RPMs each have specific usage. Table~\ref{table:rpm} outlines what each of the RPMs are used for.
\newcommand{\rpm}[2]{\code{#1} & #2\\\hline}
	\begin{center}
		\begin{table}[H]
		\caption {RPM Descriptions}
		\label{table:rpm}
			\begin{tabularx}{\textwidth}{|>{\small}l|X|}
\hline
\rowcolor{blue}\textbf{RPMs} & \textbf{Description} \\
\hline
\rpm{angryviper-ide-*.x86\_64.rpm}{The ANGRYVIPER IDE (Eclipse with plugins). See Appendix~\ref{App:IDE_plugin} for an alternative method to set up the IDE using an existing Eclipse installation.}

\rpm{opencpi-*.x86\_64.rpm}{Base installation RPM includes the runtime portion of the Component Development Kit (CDK) and the source for the \path{ocpi.core} and \path{ocpi.assets} Projects containing framework essential components, workers, platforms, etc.}

\rpm{opencpi-debuginfo-*.x86\_64.rpm}{Debug symbols needed to debug the framework.}

\rpm{opencpi-devel-*.x86\_64.rpm}{Additional header files and scripts for developing new assets as HDL and/or RCC.}

\rpm{opencpi-doc-*.x86\_64.rpm}{Includes most of the documentation found at \href{https://opencpi.github.io/}{github.io}. A symlink can be found at \code{/opt/opencpi/documentation.html}. If you receive the RPM directly from the AV team, it may include BSP documentation that is not available on GitHub.}

\rpm{opencpi-driver-*.noarch.rpm}{OpenCPI driver. Once installed, any subsequent kernel updates will cause the driver to be built automatically on restart.}

\rpm{opencpi-hw-platform-X-Y-*.noarch.rpm}{Additional files necessary to build the framework targeting specific hardware platform ``X'' when running RCC platform ``Y'' (``Y'' \textit{can} be ``\code{no\_sw}''). This RPM also includes hardware-specific SD Card images when applicable.}

\rpm{opencpi-project-bsp-*.noarch.rpm}{A \texttt{*.bsp.*} Project (\textit{e.g.} \code{ocpi.bsp.e3xx}) contains a Board Support Package for a particular physical radio, e.g. RCC/HDL Platform Support, Device Workers, etc. There are certain BSPs which are located in the \code{ocpi.assets} Project and therefore do not require their own separate BSP RPMs. As noted in Table~\ref{table:decide}, these RPMs are \textit{only} needed for development; the \code{hw-platform} RPMs contain all required runtime files to deploy to an SD Card.}

\rpm{opencpi-sw-platform-*.noarch.rpm}{Additional files necessary to build the framework targeting specific RCC/software platforms, independent of the final deployed hardware.}
			\end{tabularx}
		\end{table}
	\end{center}

\subsubsection{Installing from \code{github.io}}
Installation may be completed\footnote{This will not include the ANGRYVIPER IDE; see Appendix~\ref{App:IDE_plugin} for installation} using the publicly available RPMs on \href{https://opencpi.github.io/}{github.io} with the following reference commands:
% To test:
% docker run -it --rm centos:7
\subsubsection*{Configure Yum Repository}
\code {\$ sudo yum install yum-utils}\\
\code {\$ sudo yum-config-manager --add-repo=https://opencpi.github.io/repo/opencpi-v1.4.0.repo} % Previously did not have "0" at end; there is a symlink on github.io for either to work

\subsubsection*{Install External Dependencies}
\code {\$ sudo yum install epel-release}\\
\code {\$ sudo yum install libXft.i686 libXext.i686}\footnote{Only required on some point releases of CentOS7; others will autodetect} % AV-4733

\subsubsection*{List Available Packages}
To see packages available in the repository (to cross-reference with Tables~\ref{table:decide} and \ref{table:rpm}):\\
\code {\$ yum list 'opencpi*'}

\subsubsection*{Install}
You can choose individual packages to install (cf. Tables~\ref{table:decide} and \ref{table:rpm}), or install \textit{every} OpenCPI package available:\\
\code {\$ sudo yum install 'opencpi-*'}

\subsubsection{Installing from DVD-R}
Installation may be completed using yum with the following command:\\
\code {\$ sudo yum localinstall --nogpgcheck <location of RPMs>/*rpm}

\subsubsection{Log Out}
To enable the various \code{ocpi*} commands and set other variables, the user must log out and log back in. \textit{Opening a new terminal session is not sufficient.} This step \textit{can} be delayed until after Section~\ref{sec:setup_opencpi} is complete.

\section{Setting up the OpenCPI Environment}
\label{sec:setup_opencpi}
\subsection{HDL Simulator(s) and/or Compiler(s)}
\label{subsec:installing_fpga_vendor}
For FPGA development and/or HDL simulation, OpenCPI requires vendor-provided tools (\textit{e.g.} Xilinx Vivado, Mentor Graphics ModelSim). Refer to the \textit{FPGA Vendor Tools Installation Guide} from Table~\ref{table:references} for instruction in installing and configuring these tools for use with OpenCPI.\\

Keep note of where the \textit{license files} are, the \textit{version number} of the tools, and \textit{where the tools are installed}, as this information will be needed to configure the required environment variables.
\subsection{The \code{opencpi} Group}
\label{subsec:opencpi_group}
At this point, certain users should be added to the \code{opencpi} group. When a user creates a Project, it is likely that the Project should be \code{registered}. Registering a Project allows other users and Projects to access its assets. The default Registry on an RPM-configured system is located at \verb+/opt/opencpi/project-registry+. In order for a user to register Projects in this default location, the user will need to be a member of the \code{opencpi} group. To add a user to the \code{opencpi} group, run the following command:\\

\verb+ % sudo usermod -aG opencpi <username>+\\

If this command is run as user \verb+<username>+, the user will need to log out and back in to apply this change.\\

Note that users can use a personal non-default Project Registry. For more information on this, please visit the \textit{OpenCPI Component Development} document or the \textit{Getting Started Guide} (cf. Table~\ref{table:references}).

\subsection{Setup Environment} \label{setenv}
\label{subsec:setup_environment}
\begin{center}
\framebox{\parbox{0.8\linewidth}{\centering The Framework tries very hard to accept vendor default installation and configuration without additional settings. This section is only required if Section~\ref{subsec:installing_fpga_vendor} and/or the \textit{FPGA Vendor Tools Installation Guide} required a non-standard configuration.}}
\end{center}

Setting up the environment when installing from RPM requires root privileges. Navigate to \verb+$(OCPI_CDK_DIR)/env.d+ and notice the following example scripts:

\begin{itemize}
 	\item \verb+altera.sh.example+
 	\item \verb+modelsim.sh.example+
 	\item \verb+site.sh.example+
 	\item \verb+xilinx.sh.example+
\end{itemize}

Every time a new \code{bash}\footnote{Some problems have been reported when the user's shell is set to \path{/bin/sh} and not \path{/bin/bash}.} \textit{login} shell is opened, all \verb+*.sh+ files in \verb+/opt/opencpi/cdk/env.d+ are executed, and all \verb+*.sh.example+ files in \verb+/opt/opencpi/cdk/env.d+ are \textit{ignored}. To enable a script for execution, the name of the script must be changed so that the \verb+.example+ suffix is removed. A simple demonstration is below:\\

\verb+ % sudo cp altera.sh.example altera.sh+\\

Now \verb+altera.sh+ will execute every time a new shell is opened.\\

If using the Altera tools, the \verb+altera.sh+ will need to be created and the variables \path{OCPI_ALTERA_DIR}, \path{OCPI_ALTERA_VERSION}, and \path{OCPI_ALTERA_LICENSE_FILE} must be defined in \verb+altera.sh+. The \verb+altera.sh+ script also calls another script to set up the rest of the variables needed for the Altera tools.\\

If using the ModelSim tools, the \verb+modelsim.sh+ will need to be created and the variables \verb+OCPI_MODELSIM_DIR+ and \verb+OCPI_MODELSIM_LICENSE_FILE+ must be defined in \verb+modelsim.sh+.\\

If using the Xilinx tools, the \verb+xilinx.sh+ will need to be created and the variable \verb+OCPI_XILINX_LICENSE_FILE+ must be defined in \verb+xilinx.sh+. If using an installation of Xilinx Vivado that was \textit{not} installed in the default \verb+/opt+ directory then the variable \verb+OCPI_XILINX_VIVADO_DIR+ must be defined in \verb+xilinx.sh+. If using a version other than the most recent one installed in that location, then the variable \verb+OCPI_XILINX_VIVADO_VERSION+ must be defined in \verb+xilinx.sh+. If using an installation of Xilinx ISE that was \textit{not} installed in the default \verb+/opt+ directory then the variable \verb+OCPI_XILINX_DIR+ must be defined in \verb+xilinx.sh+. If not using the 14.7 version of ISE, then the variable \verb+OCPI_XILINX_VERSION+ must be defined in \verb+xilinx.sh+. The \verb+xilinx.sh+ script also calls another script to set up the rest of the variables needed for the Xilinx tools. See the \textit{FPGA Vendor Tools Installation Guide} (cf. Table~\ref{table:references}) for more information on Xilinx license setup.\\

The script \verb+site.sh.example+ has been provided as an example central location where any other variables can be defined globally. \textit{Remember that the names of the scripts do not matter; only the \texttt{*.sh} extension.} More configuration variables can be found in the \textit{Getting Started Guide}.\\

Once all the desired scripts have been created and edited, log out and back in and check to see that the environment is now set up.

\subsection{Removing OpenCPI RPMs}
In the event that the OpenCPI RPM needs to be uninstalled, or reinstalled, the best way to remove the OpenCPI RPM is to use \verb+yum+ to erase the RPMs from Table~\ref{table:rpm} as seen below:\\

	\verb+% sudo yum erase <RPM name>+

\section{Testing the OpenCPI installation}
\label{sec:testing_opencpi}
To verify the OpenCPI installation, there is a command \code{ocpitest} that presents various test options. \code{ocpitest --showtests} will list all available. Some require additional files to be present or Projects to be built, but for a fresh RPM install, you can use:\\

\verb+% ocpitest driver os datatype load-drivers container+\\

The first test, \code{driver}, will require \code{sudo} access. A successful install will output ``All tests passed.'' at the end of the test.

\newpage
\begin{appendices}
\appendix
\section{Prerequisites and Their Modifications}
This section provides a list of various Free and Open Source software required by the Framework. They are included within the RPMs behind the scenes to allow the Framework to function, as well as provide utility to RCC Workers. OpenCPI's packaging of these ensures they will not conflict with other\footnote{OS vendor, EPEL, other third-party-packagers, etc.} installed copies by using a nonstandard installation location. Listed below are any explicit modifications required, but implied with every item are possible modifications to the build configuration to override the library's final installation location along with
cross-compilation targeting various platforms.

% NOTE: If another prereq is ever added here, update refs that are using ad9361 as "first Appendix item"
\subsection{Analog Devices' AD9361 no-OS Library}
\label{App:ad9361}
Source: \url{https://github.com/analogdevicesinc/no-OS.git} (tied to specific git hash)\\
Diff: \path{projects/assets/prerequisites/ad9361/ad9361.patch}
\begin{enumerate}
\item[$\bullet$] Patches to allow older compilers to compile (missing \path{stdint.h} includes)
\item[$\bullet$] Fix memory leaks on de-allocation
\item[$\bullet$] Move some top-level structs from \path{common.h} into \path{ad9361.h} to limit scope of items, \textit{e.g.} ``\code{struct clk}''
\end{enumerate}

\subsection{GNU Multiple Precision Arithmetic Library (GMP)}
\label{App:gmp}
Source: \url{https://mirror.csclub.uwaterloo.ca/gnu/gmp/gmp-6.1.2.tar.xz}\\
Diff: (N/A)

\subsection{Google Test (GTEST)}
\label{App:gtest}
Source: \url{https://github.com/google/googletest/archive/release-1.8.0.zip}\\
Diff: (N/A)\\
Note: Not available to RCC Workers

\subsection{Liquid DSP}
Source: \url{https://github.com/jgaeddert/liquid-dsp/archive/v1.3.1.tar.gz}\\
Diff: \path{projects/assets/prerequisites/liquid/malloc.patch}
\label{App:liquid}
\begin{enumerate}
\item[$\bullet$] Disable \path{autoconf} macros for \path{malloc()} and \path{realloc()} that cause missing \path{rpl_malloc} symbols for some cross-compilers; assumes cross-compilers have sane/modern \path{malloc()} implementations
\end{enumerate}

\subsection{Lempel-Ziv-Markov chain algorithm (LZMA/XZ)}
Source: \url{https://tukaani.org/xz/xz-5.2.3.tar.gz}\\
Diff: (N/A)
\label{App:xz}

\subsection{PatchELF}
Source: \url{http://nixos.org/releases/patchelf/patchelf-0.9/patchelf-0.9.tar.gz}\\
Diff: (N/A)\\
Note: Not available to RCC Workers
\label{App:patchelf}

% NOTE: If another prereq is ever added here, update refs that are using patchelf as "last Appendix item"

\newpage
\section{Appendix - Add the ANGRYVIPER IDE to Eclipse using the Plugin}
\label{App:IDE_plugin}
The ANGRYVIPER IDE is constructed using the Eclipse Neon release and a plugin developed by the ANGRYVIPER team. Since the entire IDE is too large to be placed on GitHub, the following instructions may be used to obtain the IDE by downloading Eclipse and installing the plugin as an Eclipse drop-in.

\subsubsection*{Download Plugin JAR}
\begin{enumerate}
\item Obtain the latest ANGRYVIPER plugin jar file
\subitem \verb+wget https://opencpi.github.io/ide/av.proj.ide.plugin_1.4.jar+
\end{enumerate}

\subsection{Eclipse (Neon Release)}
\begin{enumerate}
\item Download the Eclipse Neon IDE for C/C++ Developers
\subitem URL: https://www.eclipse.org/neon/
\item Install Eclipse by extracting the archive in the desired location
\item Start Eclipse
\subitem Go into the folder where it was installed and click/run \path{eclipse}
\item Put the \path{av.proj.ide.plugin_*.jar} file in the \path{eclipse/dropins} folder
\item Install Sapphire via the Eclipse Marketplace
\subitem In Eclipse, navigate to ``Help $\rightarrow$ Eclipse Marketplace''. Search for ``Sapphire''. There should be one search result for Sapphire.
Click the ``Install'' button.  Sapphire and its dependencies will be installed.
\item Restart Eclipse when prompted.
\item Eclipse now has the ANGRYVIPER IDE functionality.
\end{enumerate}

\subsection{Eclipse (Oxygen Release)}
The process to construct the IDE is the same as described above using the Oxygen release for C/C++ Developers. \\

\textit{Note}: At this time, the ANGRYVIPER Team has not been able to 100\% verify using the plugin in Oxygen release. Eclipse Oxygen changed an API that caused problems for Sapphire, and Sapphire 9.1.1 has been released to correct the issue. The unknown part of the process is whether or not the Eclipse Marketplace will have the new version of Sapphire. If it does not, it can be installed manually as follows:
\begin{enumerate}
\item In Eclipse, navigate to ``Help $\rightarrow$ Install New Software''.
\item Add the Sapphire 9.1.1 repository
\subitem Click the ``add'' button (to add a new repository site), fill in the popup form:
\subitem \code{name: Sapphire9.1.1}
\subitem \code{location: http://download.eclipse.org/sapphire/9.1.1/repository/}
\item Click ``OK'' to add it
\item Select the down arrow at the end of the ``work with:'' input. Select the new Sapphire repository.
\item Select Sapphire. If Samples and Tests appear in the list; deselect them.
\item Install
\end{enumerate}

\end{appendices}
\end{document}
