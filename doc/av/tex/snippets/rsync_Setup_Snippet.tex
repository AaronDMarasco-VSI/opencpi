\iffalse
This file is protected by Copyright. Please refer to the COPYRIGHT file
distributed with this source distribution.

This file is part of OpenCPI <http://www.opencpi.org>

OpenCPI is free software: you can redistribute it and/or modify it under the
terms of the GNU Lesser General Public License as published by the Free Software
Foundation, either version 3 of the License, or (at your option) any later
version.

OpenCPI is distributed in the hope that it will be useful, but WITHOUT ANY
WARRANTY; without even the implied warranty of MERCHANTABILITY or FITNESS FOR A
PARTICULAR PURPOSE. See the GNU Lesser General Public License for more details.

You should have received a copy of the GNU Lesser General Public License along
with this program. If not, see <http://www.gnu.org/licenses/>.
\fi

% This is for inserting into various "Getting Started" Guides
% First, turn off indenting to avoid all the flushleft
\newlength{\savedparindentrsync}%
\setlength{\savedparindentrsync}{\parindent}%
\setlength{\parindent}{0pt} % Don't indent all paragraphs
\providecommand{\forceindent}{\leavevmode{\parindent=1em\indent}}%
\def\qrsync{``\code{rsync}''~}
\subsection{\qrsync provided binary}
\label{sec:rsync}
An ARM-compiled version of \qrsync is provided in the included SD card image for \rccplatform.
This tool allows the use of \textit{standalone mode} while shortening the required developer time to synchronize the artifacts being developed.
For command-line usage, see the \href{https://rsync.samba.org/documentation.html}{rsync home page}.
The easiest usage is to have the radio ``pull'' from the developer's workstation; this does not need any additional command-line arguments.
\subsubsection*{Implementation Details}
Unfortunately, the \qrsync executable is not in the default path because when called remotely, it requests a non-interactive shell. For this reason, a ``pull'' approach is recommended.
If a user for some reason requires a ``push'' from the workstation to the radio, the local \qrsync executable must be told the \textit{remote location} of the \path{rsync} executable to call, \textit{e.g.} \code{rsync --rsync-path=/mnt/card/opencpi/\rccplatform/bin/rsync}
\setlength{\parindent}{\savedparindentrsync}%
