\section{Driver Setup}
\begin{flushleft}
Currently, OpenCPI PCIe platforms require 128MB of contiguous RAM memory for accessing the control plane (1 MB * 64 workers) and data (32MB send + 32MB receive). By default, the Linux kernel reserves only 128KB of RAM per devices. Therefore, the Linux kernel boot must be modified to reserve a larger block memory for OpenCPI purposes. \\ \medskip
The memmap parameter is used to reserved more block memory from the Linux kernel. It takes a number of formats, but the following usage has proven to be sufficient: \\
\bigskip
	memmap=SIZE\$START \\
\bigskip
Where SIZE is the number of bytes to reserve in either hex or decimal, and
START is the physical address in hexadecimal bytes. It is required that the pages for all addresses and sizes are on even boundaries (0x1000 or 4096 bytes). \\
\subsection{Calculate Values in Preparation for Memory Reservation}
Run dmesg and filter on BIOS to review the physical RAM map: \\
\lstset{language=bash, backgroundcolor=\color{lightgray}, columns=flexible, breaklines=true, prebreak=\textbackslash, basicstyle=\ttfamily, showstringspaces=false,upquote=true, aboveskip=\baselineskip, belowskip=\baselineskip}
\begin{lstlisting}
dmesg | grep BIOS
\end{lstlisting}
The output will look something like:
\begin{lstlisting}
BIOS-provided physical RAM map:
 BIOS-e820: 0000000000000000 - 000000000009f800 (usable)
 BIOS-e820: 000000000009f800 - 00000000000a0000 (reserved)
 BIOS-e820: 00000000000ca000 - 00000000000cc000 (reserved)
 BIOS-e820: 00000000000dc000 - 00000000000e4000 (reserved)
 BIOS-e820: 00000000000e8000 - 0000000000100000 (reserved)
 BIOS-e820: 0000000000100000 - 000000005fef0000 (usable)
 BIOS-e820: 000000005fef0000 - 000000005feff000 (ACPI data)
 BIOS-e820: 000000005feff000 - 000000005ff00000 (ACPI NVS)
 BIOS-e820: 000000005ff00000 - 0000000060000000 (usable)
 BIOS-e820: 00000000e0000000 - 00000000f0000000 (reserved)
 BIOS-e820: 00000000fec00000 - 00000000fec10000 (reserved)
 BIOS-e820: 00000000fee00000 - 00000000fee01000 (reserved)
 BIOS-e820: 00000000fffe0000 - 0000000100000000 (reserved)
\end{lstlisting}
Select a "(usable)" section of memory and reserve a subsection of that memory. Once the memory is reserved, the Linux kernel will ignore it.  In
this example, there are three usable sections:\\
\begin{lstlisting}
 BIOS-e820: 0000000000000000 - 000000000009f800 (usable)
 BIOS-e820: 0000000000100000 - 000000005fef0000 (usable)
 BIOS-e820: 000000005ff00000 - 0000000060000000 (usable)
\end{lstlisting}
Due to the way Linux manages memory, it is recommended that addresses above the first 24 bits be reserved for OpenCPI. Thus, for this example, the best choice is the second section (pages 0x100000-0x5FEF0000).  \\ \medskip
Given the (default) requirement to have 128MB of contiguous memory and the addresses must be on an even page boundary of 4KB:\\ \medskip 128MB/(4KB/page) = 134217728/4096 = 32768, or 0x8000 pages of contiguous memory. \\ \medskip

To determine the start of the new reserve address, subtract the
required number of pages from the end of the block, minus an additional buffer (0x5FEF*0x10=0x5FEF0) for extra margin. \\ \medskip
0x5FEF0-0x8000=0x57EF0. \\ \medskip

Adjusting for the actual physical RAM map provides 0x57EF0000, which is used to construct the memmap parameter:\\ \medskip
memmap=128M\$0x57EF0000\\

\subsection{Configure Memory Reservation}
\textbf{Critical Note:
If other memmap parameters are implemented, e.g. for non-OpenCPI PCI cards, then grubby usage will be different. The OpenCPI driver will use the first memmap parameter on the command line OR the parameter ``opencpi\_memmap'' if it is explicitly given. If this parameter is given, the standard memmap command with the same parameters must ALSO be passed to the kernel.}\\ \bigskip

Once the memmap parameter as been calculated, it will need to be added to the kernel command line in the boot loader. \\
\bigskip
For CentOS, the utility ``grubby'' can be used to add the parameter to all kernels in the start-up menu. The single quotes are REQUIRED or the shell will interpret the \$0: \\
\bigskip
\textbf{\textit{CentOS6}}:\\
\begin{lstlisting}
sudo grubby --update-kernel=ALL --args=memmap='128M\$0x57EF0000'
\end{lstlisting}
\textbf{\textit{CentOS 7}} uses \textit{grub2}, which \textbf{requires a DOUBLE} backslash:\\
\begin{lstlisting}
sudo grubby --update-kernel=ALL --args=memmap='128M\\$0x57EF0000'
\end{lstlisting}

To verify the current kernel has the argument set:\\
\begin{lstlisting}
sudo -v
sudo grubby --info $(sudo grubby --default-kernel)
\end{lstlisting}

\textbf{\textit{CentOS 7}} displays a \textbf{SINGLE} backslash before the \$, for example: \\
\begin{lstlisting}
args="ro rdblacklist=nouveau crashkernel=auto rd.lvm.lv=vg.0/root quiet audit=1 boot=UUID=96933cb5-f478-4933-a0d4-16953cf47f5c memmap=128M\$0x57EF0000 LANG=en_US.UTF-8"
\end{lstlisting}

If no longer desired, the parameter can also be removed:
\begin{lstlisting}
sudo grubby --update-kernel=ALL --remove-args=memmap
\end{lstlisting}

More information concerning grubby can be found at:\\
\url{https://access.redhat.com/documentation/en-US/Red_Hat_Enterprise_Linux/7/html/System_Administrators_Guide/sec-Making_Persistent_Changes_to_a_GRUB_2_Menu_Using_the_grubby_Tool.html}
\bigskip

For the memmap parameter:\\
\url{https://www.kernel.org/doc/html/latest/admin-guide/kernel-parameters.html}

\subsection{Apply Memory Reservation}
Reboot the system, making certain to boot from the new configuration.
\subsection{Verify Memory Reservation}
Once the system has finished booting, examine the state of the physical RAM map to confirm that the desired memory has been reserved:\\
\bigskip
\begin{lstlisting}
dmesg | more
Linux version 2.6.18-128.el5 (mockbuild@hs20-bc1-7.build.redhat.com) (gcc version 4.1.2 20080704 (Red Hat 4.1.2-44)) #1 SMP Wed Dec 17 11:41:38 EST 2008
Command line: ro root=/dev/VolGroup00/LogVol00 rhgb quiet memmap=128M$0x57EF0000
BIOS-provided physical RAM map:
 BIOS-e820: 0000000000000000 - 000000000009f800 (usable)
 BIOS-e820: 000000000009f800 - 00000000000a0000 (reserved)
 BIOS-e820: 00000000000ca000 - 00000000000cc000 (reserved)
 BIOS-e820: 00000000000dc000 - 00000000000e4000 (reserved)
 BIOS-e820: 00000000000e8000 - 0000000000100000 (reserved)
 BIOS-e820: 0000000000100000 - 000000005fef0000 (usable)
 BIOS-e820: 000000005fef0000 - 000000005feff000 (ACPI data)
 BIOS-e820: 000000005feff000 - 000000005ff00000 (ACPI NVS)
 BIOS-e820: 000000005ff00000 - 0000000060000000 (usable)
 BIOS-e820: 00000000e0000000 - 00000000f0000000 (reserved)
 BIOS-e820: 00000000fec00000 - 00000000fec10000 (reserved)
 BIOS-e820: 00000000fee00000 - 00000000fee01000 (reserved)
 BIOS-e820: 00000000fffe0000 - 0000000100000000 (reserved)
user-defined physical RAM map:
 user: 0000000000000000 - 000000000009f800 (usable)
 user: 000000000009f800 - 00000000000a0000 (reserved)
 user: 00000000000ca000 - 00000000000cc000 (reserved)
 user: 00000000000dc000 - 00000000000e4000 (reserved)
 user: 00000000000e8000 - 0000000000100000 (reserved)
 user: 0000000000100000 - 0000000057ef0000 (usable)
 user: 0000000057ef0000 - 000000005fef0000 (reserved)  <== New
 user: 000000005fef0000 - 000000005feff000 (ACPI data)
 user: 000000005feff000 - 000000005ff00000 (ACPI NVS)
 user: 000000005ff00000 - 0000000060000000 (usable)
 user: 00000000e0000000 - 00000000f0000000 (reserved)
 user: 00000000fec00000 - 00000000fec10000 (reserved)
 user: 00000000fee00000 - 00000000fee01000 (reserved)
 user: 00000000fffe0000 - 0000000100000000 (reserved)
DMI present.
\end{lstlisting}

A new "(reserved)" area is shown between the second "(useable)" section and the (ACPI data) section. Now, when the "ocpidriver load" is ran, it will detect the new reserved area, and pass that data to the OpenCPI kernel module. \\
\end{flushleft}
